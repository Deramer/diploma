\documentclass[a4paper,12pt]{article}

%%	Fonts
\usepackage{cmap}
\usepackage{mathtext}
\usepackage{hyperref}
\usepackage[T2A]{fontenc}
\usepackage[utf8]{inputenc}
\usepackage[english,russian]{babel}
\usepackage[scaled]{helvet}
%\usepackage{fullpage}

% 	Math packages
\usepackage{amsmath,amsfonts,amssymb,amsthm,mathtools}
\usepackage{icomma}
\usepackage{euscript}
\usepackage{mathrsfs}
\usepackage{tensor}
\usepackage{physics}
\usepackage{stackrel}
% 	My math operartors
%\mathtoolsset{showonlyrefs=true}
\newcommand{\bigo}[1]   {\,\ensuremath{\mathop{}\mathopen{}\mathcal{O}\mathopen{}\displaystyle\left(#1\right)}}
\newcommand{\smallo}[1]	{\,\scriptstyle\mathop{}\mathopen{}\mathcal{O}\mathopen{}\displaystyle\left(#1\right) }
\DeclareMathOperator{\arcsinh}{arcsinh}
\DeclareMathOperator{\arctanh}{arctanh}
\newcommand*{\hm}[1]{#1\nobreak\discretionary{} {\hbox{$\mathsurround=0pt #1$}}{}}

%%	Graphics
\usepackage{graphicx}
\graphicspath{{images/}}
\setlength\fboxsep{3pt}
\setlength\fboxrule{1pt}
\usepackage{wrapfig}
% 	Tables
\usepackage{array,tabularx,tabulary,booktabs}
\usepackage{longtable}
\usepackage{multirow}
\usepackage{caption}
%\usepackage{subcaption}
\usepackage{subfig}
%	Itemize
\usepackage{enumitem}
\setlist[itemize]{topsep=0pt, leftmargin=0.2in}

%% 	Theorems
\theoremstyle{plain} % Это стиль по умолчанию, его можно не переопределять.
 \newtheorem{theorem}{Теорема}[section]
 \newtheorem{proposition}{Утверждение}
%\newtheorem{proposition}[theorem]{Утверждение}
 \newtheorem{lemma}{Лемма}
\theoremstyle{definition} % "Определение"
 \newtheorem{definition}{Определение}[section]
 \newtheorem{corollary}{Следствие}[theorem]
 \newtheorem{problem}{Задача}[section]
\theoremstyle{remark} % "Примечание"
 \newtheorem*{solution}{Решение}
\renewcommand\qedsymbol{$\blacksquare$}
\newcommand{\proofbegin}{\ensuremath{\blacktriangle}\nopunct}
\newcommand{\contbegin}{\footnotesize{\textcircled{\scriptsize!}}\normalsize \ }
\newcommand{\contend}{\ensuremath{\otimes}}
\newcommand{\vect}[1]{\ensuremath{\overrightarrow{#1}}}

%%	Programming
\usepackage{etoolbox} % логические операторы

%%	Page
\usepackage[14pt]{extsizes} % Возможность сделать 14-й шрифт
\usepackage{geometry} % Простой способ задавать поля
	\geometry{top=20mm}
	\geometry{bottom=20mm}
	\geometry{left=22mm}
	\geometry{right=22mm}
	\geometry{bindingoffset=0mm}
%\linespread{0.01}
%\usepackage{mathpazo}
%usepackage{fancyhdr} % Колонтитулы
% 	\pagestyle{fancy}
% 	\renewcommand{\headrulewidth}{.1mm}  % Толщина линейки, отчеркивающей верхний колонтитул
% 	\lfoot{}
% 	\rfoot{}
% 	\rhead{}
% 	\chead{}
% 	\lhead{}
 	% \cfoot{Нижний в центре} % По умолчанию здесь номер страницы
\usepackage{setspace} % Интерлиньяж
\onehalfspacing % Интерлиньяж 1.5
%\doublespacing % Интерлиньяж 2
%\singlespacing % Интерлиньяж 1
%\linespread{0.01}
\usepackage{lastpage} % Узнать, сколько всего страниц в документе.
\usepackage{soul} % Модификаторы начертания
%\usepackage{hyperref}
\usepackage[usenames,dvipsnames,svgnames,table,rgb]{xcolor}
\hypersetup{					% Гиперссылки
    unicode=true,           	% русские буквы в раздела PDF
    pdftitle={},   				% Заголовок
    pdfauthor={Nestyuk Arseniy},	% Автор
    pdfsubject={},      		% Тема
    pdfcreator={Nestyuk Arseniy}, 	% Создатель
    pdfproducer={}, 			% Производитель
    pdfkeywords={Qubit} {Quantum physics} {Majorana}, % Ключевые слова
    colorlinks=true,       		% false: ссылки в рамках; true: цветные ссылки
    linkcolor=red,          	% внутренние ссылки
    citecolor=green,        	% на библиографию
    filecolor=magenta,      	% на файлы
    urlcolor=cyan           	% на URL
}
\usepackage{multicol} % Несколько колонок

% 	Titel
\author{}
\title{}
\date{\today}

\graphicspath{ {images/} }


\begin{document}
    
\begin{titlepage}
    \begin{center}
        Министерство образования и науки Российской Федерации
        \linebreak
        
        Федеральное государственное автономное
        
        образовательное учреждение высшего образования 
        
        «Московский физико-технический институт 
        
        (государственный университет)»
        \linebreak
        
        Факультет общей и прикладной физики
        
        Кафедра "Проблемы теоретической физики"
        \vspace{100pt}
        
        \textbf{\LARGE Майорановские кубиты в джозефсоновских изинговских цепочках}
        \linebreak
        
        Выпускная квалификационная работа бакалавра
     \end{center}
     \vspace{90pt}
     
     \begin{flushright}
         \begin{tabular}{cl}
         &Выполнил:\\
         &студент 322 группы\\         
         &А.П. Нестюк\\
         &\\
         &Научный руководитель:\\
         &д.ф-м.н., проф. Ю.Г. Махлин
        \end{tabular}
    \end{flushright}
    \vspace{60pt}
    
    \begin{center}
        Черноголовка, 2017
    \end{center}
 \end{titlepage}   

\tableofcontents
\pagebreak

\section{Введение}
В последнее время большой интерес вызывают возможности реализации кубитов на основе так называемых майорановских нулевых мод. Реализация таких кубитов и системы управления ими может оказаться ключевым фактором в построении квантового компьютера, состоящего из большого числа кубитов.

Впервые майорановские нулевые моды были рассмотрены Китаевым в \cite{kitaev}. Он рассмотрел фермионную цепочку и предложил описывать её на языке майорановских фермионов. В пространстве коэффициентов гамильтониана этой системы есть область, в которой основное состояние двукратно вырождено. Существует также фермионный оператор, переводящий одно из основных состояний в другое. Он состоит из операторов с разных концов цепочки (все остальные входят в него с экспоненциально малыми множителями) - фактически, он локализован на концах цепочки. Из-за этого физические, локальные возмущения гамильтониана, сохраняющие фермионную чётность, расщепляют основное состояние экспоненциально слабо по длине цепочки. Потеря когерентности из-за влияния шумов является одной из основных проблем создания кубитов, и данная система практически полностью решает этот вопрос.

К сожалению, решая один вопрос, майорановские нулевые моды ставят новые. Во-первых, их техническая реализация ещё далека от совершенства. Одной из возможных основ для создания таких систем сейчас считаются p-волновые сверхпроводники, и для них получены только признаки существования таких фермионов (обзор и анализ экспериментов приведены, например, в \cite{computation}). Об управлении ими речь пока не идёт.

Во-вторых, строго доказано, что нельзя выполнить произвольный квантовый алгоритм, не нарушив защиту майорановских нулевых мод (см., например, \cite{anyons}), хотя и были предложены операции, защиту не нарушающие. В науке о квантовых алгоритмах вводится понятие ``универсального набора операций'' - набора, достаточного для осуществления любой унитарной операции над системой, то есть любого квантового алгоритма. Однако над майорановскими нулевыми модами реализовать универсальный набор операций можно только нарушив топологическую защиту.

Тем не менее, интерес к майорановским нулевым модам не угас. Технические сложности обычно преодолимы. Кроме того, такие кубиты можно будет использовать в качестве долговременной памяти, поскольку без проведения над ними операций они сохраняют своё состояние достаточно долго.

Более того, появился интерес к исследованиям, связанным с прочими системами, которые можно описывать на языке майорановских фермионов. Одной из таких систем является спиновая цепочка. К фермионному языку её сводит преобразование Йордана-Вигнера. Получившуюся фермионную систему нетрудно свести к системе майорановских фермионов. Однако, поскольку преобразования Йордана-Вигнера нелокально, обычные локализованные шумы спиновой системы становятся нелокальными шумами в фермионной системе, из-за чего теряется защита от шумов - основная причина интереса. Тем не менее, поскольку без разрушения защиты невозможно выполнять квантовые алгоритмы, эта проблема является неизбежной при построении полноценного кубита. Поскольку спиновые системы на основе джозефсоновских кубитов изучены относительно хорошо, а времена их когерентности позволяют проводить с ними эксперименты, можно уже сейчас проверить, как соотносится теория с практикой. Кроме того, возможно, что майорановские кубиты на сверхпроводниках будут использоваться в основном как память компьютера, а спиновые системы, полностью находящиеся под нашим контролем, - как способ проводить операции и хранить данные небольшое количество времени (процессор в терминах современных компьютеров).

В работе \cite{main} описаны теоретические основы и элементы практической реализации подобных спиновых систем. Единственная доступная операция над майорановскими нулевыми модами, не нарушающая топологической защиты, требует выхода за пределы одномерной цепочки, что представляет дополнительные сложности в момент выполнения преобразования Йордана-Вигнера. Дальнейшие возникающие затруднения и указанные в статье методы их устранения будут описаны ниже, в главе \ref{sec:spin_system}.

В качестве физической реализации этой спиновой системы планируется использовать сверхпроводящие потоковые кубиты на джозефсоновских контактах. Конструкция таких устройств и их времена когерентности описаны в \cite{flux_qubit}, особенности применения для системы, исследуемой в данной работе, можно найти в \cite{main}.

Целью данной работы является более подробное изучение спиновой системы, описанной в статье \cite{main}. Будут рассмотрены различные погрешности, включая влияние различных шумов на состояние кубита и фактический размер майорановских нулевых мод. Будут сделаны некоторые общие выводы относительно защищённости этой системы от случайных шумов.

В главе \ref{sec:majorana_modes} будет описана модель Китаева, в которой и возникают майорановские нулевые моды, а также про возможную операцию над ними - braiding, переплетение. Далее, в главе \ref{sec:spin_system}, будет показано, какую спиновую систему можно описывать на майорановском языке, какое преобразование нужно для этого сделать, а также как выглядит переплетение в спиновом представлении. Наконец, в последней главе \ref{sec:results} будут представлены результаты вычислений различных поправок, связанных с неидеальностями системы, описанной главе \ref{sec:spin_system}. Основное внимание будет уделено влиянию неточностей выполнения операций и возмущений, действующих на спины системы.

% Moving to next section
% Majorana modes in fermionic systems

\pagebreak

\section{Майорановские моды в фермионных системах} \label{sec:majorana_modes}

В этой главе речь пойдёт о гамильтониане Китаева и его свойствах, о майорановских нулевых модах и их топологической защите, а также о единственной операции, которую можно выполнить над этими модами, не нарушив защиту - о переплетении.

Основой для дальнейшего изложения служат статьи \cite{kitaev}, в которой вводится понятие майорановских нулевых мод, и \cite{braiding}, где предлагается методика выполнения операции над ними. 

\subsection{Модель Китаева} \label{subsec:kitaev}
Следуя \cite{kitaev}, рассмотрим фермионную цепочку со следующим гамильтонианом:
\begin{equation}
    H = \sum\limits_j -\omega \left( a_j^\dagger a_{j+1} + a_{j+1}^\dagger a_j - a_j a_{j+1} - a_{j+1}^\dagger a_j^\dagger \right) - \mu \left( a_j^\dagger a_j - \frac{1}{2} \right)
    \label{eq:main_hamiltonian_fermi}
\end{equation}
(строго говоря, в гамильтониане Китаева коэффициенты перед $a_j^\dagger a_{j+1}$ и $a_j a_{j+1}$ разные, но для удобства изложения здесь они положены одинаковыми).

Майорановскими фермионными операторами называется набор операторов, удовлетворяющий соотношениям
\begin{align}
\gamma_i^2 &= 1, & \{\gamma_i, \gamma_j\} = 2 \delta_{ij}.
\end{align}
Эти операторы можно построить и из дираковских операторов, например, следующим образом:
\begin{align}
\gamma_{2j-1} &= a_j + a_j^\dagger, & \gamma_{2j} &= i\left( a_j^\dagger - a_j \right),
\end{align}
причём $\gamma$ с чётными и нечётными индексами, как и произошедшие из разных фермионов, различить больше нельзя: их можно разбить на пары произвольным образом и сделать из каждой пары дираковский фермионный оператор.

Гамильтониан после преобразования имеет вид
\begin{equation}
H = -i\mu/2 \sum\limits_{j=1}^N \gamma_{2j-1} \gamma_{2j} + i \omega \sum\limits_{j=1}^{N-1} \gamma_{2j} \gamma_{2j+1}.
\label{eq:main_hamiltonian}
\end{equation}
Нетрудно заметить, что в предельных случаях, когда либо $\mu = 0$, либо $\omega = 0$, получаются два принципиального разных гамильтониана,
\begin{align}
H &= -i\mu/2 \sum\limits_{j=1}^N \gamma_{2j-1} \gamma_{2j}, & H &= i \omega \sum\limits_{j=1}^{N-1} \gamma_{2j} \gamma_{2j+1}.
\end{align}
Во втором отсутствуют операторы $\gamma_1$ и $\gamma_{2L}$, которые можно объединить в один дираковский фермион, коммутирующий с гамильтонианом. Основное состояние системы, таким образом, двукратно вырождено, и этот фермион переводит одно состояние в другое.

Это двукратное вырождение защищено от возмущений, причём чем длиннее цепочка, тем меньше влияние шумов (и меньше экспоненциально). Это можно показать следующим образом. Предположим сохранение чётности рассматриваемой фермионной цепочки, тогда все слагаемые гамильтониана будут чётной степени по фермионным операторам. Из квадратичных слагаемых на майорановском языке остаются только те, в которых у операторов разные индексы - квадрат любого майорановского фермиона равен $1$. Чтобы действие возмущения появилось в первом порядке теории возмущений, оно должно содержать операторы с разных концов цепочки ($1$-ого и $2L$-ого). То есть возмущение должно быть нелокально, что нефизично. 

В первом гамильтониане возбуждения с нулевой энергией нет. Как показано в работе \cite{kitaev}, когда оба коэффициента в гамильтониане отличны от нуля, наличие такого возбуждения зависит от отношения коэффициентов $\mu$ и $\omega$: когда $\mu/\omega > 2$, его нет, в противном случае есть. Причём когда возбуждение есть, оно затухает вглубь образца с экспоненциальной скоростью.

Видно, что образуются две фазы, причём в одной из них на границе образца с пустотой ``живут'' майорановские нулевые моды, также называемые краевыми модами. Такую фазу будем далее называть топологической, а другую - топологически тривиальной. Следует заметить, что если внутри образца проходит граница между двумя фазами, то майорановская нулевая мода локализована на этой границе и затухает по обе стороны от неё.


\subsection{Переплетение майорановских нулевых мод} \label{subsec:braiding}

Поскольку майорановские моды защищены от внешних воздействий, возникает проблема управления ими, а без контролируемого изменения состояния кубита невозможны квантовые вычисления.

В \cite{braiding} предлагается операция, не нарушающая защиту майорановских нулевых мод. Она называется braiding. Это слово в данной дипломной работе будет переводиться как ``переплетение''. Идея состоит в том, что если поменять местами майорановские нулевые моды, то состояние системы изменится. Это легко понять, введя оператор чётности
\begin{equation}
T = \prod\limits_{j=1}^{2L} (-i \gamma_j).
\end{equation}
Этот оператор коммутирует с гамильтонианом (\ref{eq:main_hamiltonian}). В самом деле, в гамильтониан входят только чётные по числу майорановских фермионов слагаемые, а в оператор чётности - каждый из таких фермионов один раз. Учитывая, что различные фермионы антикоммутируют, а сами с собой - коммутируют, получим
\begin{align}
\gamma_n T &= - T \gamma_n & \gamma_n \gamma_m T = T \gamma_n \gamma_m.
\end{align}

Если теперь поменять местами два фермионных оператора, оператор чётности изменит знак:
\begin{equation}
\begin{split}
T_{nm} &= (-i)^{2N} \gamma_1 \dotsm \gamma_n \dotsm \gamma_m \dotsm \gamma_{2N} \\
    &= (-i)^{2N} (-1)^{m-n-1} \gamma_1 \dotsm \gamma_n \gamma_m \dotsm \gamma_{2N} \\ &= (-i)^{2N} (-1)^{m-n} \gamma_1 \dotsm \gamma_m \gamma_n \dotsm \gamma_{2N} \\
    &= - (-i)^{2N} \gamma_1 \dotsm \gamma_m \dotsm \gamma_n \dotsm \gamma_{2N} \\
    &= - T_{mn}.
\end{split}
\end{equation}

Меняя коэффициенты гамильтониана, двумерном пространстве можно обвести точки локализации майорановских мод друг вокруг друга. Чётность при этом, очевидно, измениться не должна. При этом операторы (в гейзенберговском представлении) не могут просто переходить друг в друга: один из них меняет знак, чтобы сохранить чётность. Соответственно этому изменится и состояние системы.

В одномерной цепочке невозможно переставить фермионы, не проведя их друг через друга. В \cite{braiding} предлагается выйти из одномерного пространства, рассмотрев решётку из проводов. Но для проведения одной операции переплетения достаточно более простой системы, Т-образного соединения, которое является частью двумерной решётки. Исследуем изменения относительно предыдущего пункта поэтапно.

Сначала нужно рассмотреть фермионную цепочку с гамильтонианом \ref{eq:main_hamiltonian_fermi}, коэффициенты которого зависят от координат. Пусть зависимость отношения этих коэффициентов такова, как на рис. \ref{fig:two_phases}. Внутри каждого из участков коэффициенты гамильтониана постоянны, при этом в среднем участке отношение коэффициентов такого, что он находится в топологической фазе, а другие участки - в топологически тривиальной.

\begin{figure}
    \centering
    \begin{minipage}{.5\textwidth}
        \centering
        \includegraphics[width=\linewidth]{two_phases}
        \captionsetup{width=0.9\textwidth}
        \captionof{figure}{Зависимость отношения коэффициентов гамильтониана от номера узла; горизонтальная линия показывает границу фаз.}
        \label{fig:two_phases}
    \end{minipage}%
    \begin{minipage}{.5\textwidth}
        \centering
        \includegraphics[width=\linewidth]{T_junction}
        \captionsetup{width=0.9\textwidth}
        \captionof{figure}{T-образное соединение; точками показаны фермионы, линиями - их взаимодействие.}
        \label{fig:t_junction}
    \end{minipage}
\end{figure}

В такой системе майорановские нулевые моды будут находиться на границе фаз; при конечном отношении коэффициентов возбуждение будет экспоненциально затухать не только внутрь топологической фазы, но и ``наружу'', в топологически тривиальную фазу.

Имея возможность менять коэффициенты гамильтониана независимо в каждой точке, можно изменять их отношение так, что граница фаз будет смещаться. При адиабатическом выполнении данной операции состояние майорановских нулевых мод не будет меняться. Это означает, что мы можем двигать топологический участок в произвольное место цепочки.

Наконец, можно представить себе соединение двух фермионных цепочек. Простейший вариант, представленный на рис. \ref{fig:t_junction} и называемый T-\hspace{0pt}образным соединением, является обычной фермионной цепочкой, один из узлов которой взаимодействует с крайним фермионом другой цепочки. В такой системе тоже существуют две фазы, определяемые отношением коэффициентов гамильтониана, и на границе между ними находятся майорановские нулевые моды. Соответственно, границы между фазами в такой конструкции можно двигать в том числе и в между тремя ветвями T-\hspace{0pt}образного соединения.

\begin{figure}
    \centering
    \subfloat{%
        \centering
        \includegraphics[width=0.25\textwidth]{braidingA}}%
    \subfloat{%
        \centering
        \vspace{10pt}
        \includegraphics[width=0.25\textwidth]{braidingB}}%
    \subfloat{%
        \centering
        \includegraphics[width=0.25\textwidth]{braidingC}}%
    \subfloat{%
        \centering
        \includegraphics[width=0.25\textwidth]{braidingD}}%
    \caption{Braiding, переплетение майорановских мод в Т-образно соединённых фермионных цепочках, \cite{braiding}.}
    \label{fig:braiding}
\end{figure}

В таком T-\hspace{0pt}образном соединении можно переставлять майорановские моды. Процесс представлен на рис. \ref{fig:braiding}. В вышеописанной системе одна из мод уводится в нижнее ответвление, другая переводится на исходное место первой, после чего первая передвигается туда, где в начале была вторая. Все процессы осуществляются адиабатически и с сохранением чётности.

Переплетение важно, поскольку не нарушает топологической защиты и является частью универсального набора операций, упомянутого во введении. Переплетение двух майорановских мод, относящихся к разным кубитам, является нетривиальной операцией над кубитами. Кроме того, переплетение мод, относящихся к одному кубиту, меняет его состояние. Однако эта операция должна быть дополнена, поскольку сама по себе не составляет универсального набора операций.

% Moving to next section
% Spin system

\pagebreak

\section{Спиновая система} \label{sec:spin_system}

Изложение в данной части основано на \cite{main}. Будет показано, как можно совершить переход от спиновой системы к фермионной, с учётом того факта, что цепочка не является одномерной и обычное преобразование Йордана-Вигнера не даёт гамильтониана с взаимодействием ближайших соседей. Далее будет рассмотрено переплетение на спиновом языке.

\subsection{Модифицированное преобразование Йордана-Вигнера} \label{subsec:jordan}

Обычное преобразование Йордана-Вигнера (здесь переводящее спиновую систему на язык майорановских фермионов),
\begin{align}
\gamma_{2n-1} &= \sigma^z (n) \sum\limits_{p=1}^{n-1} \sigma^x (p), & 
\gamma_{2n} &= \sigma^y (n) \sum\limits_{p=1}^{n-1} \sigma^y (p),
\end{align}
превращает изинговскую цепочку с поперечным магнитным полем
\begin{equation}
H = - J \sum\limits_{n=1}^{N-1} \sigma^z(n) \sigma^z(n+1) - \sum\limits_{n=1}^N h(n) \sigma^x(n)
\label{eq:spin_H}
\end{equation}
в майорановскую цепочку с квадратичным гамильтонианом и взаимодействием ближайших соседей
\begin{equation}
H = -i \sum\limits_{n=1}^N h(n) \gamma(2n-1) \gamma(2n) + i J \sum\limits_{n=1}^{N-1} \gamma(2n) \gamma(2n+1).
\end{equation}

Для этого преобразования важно уметь последовательно перенумеровывать спины. В случае, когда есть три ветки (рис. \ref{fig:t_junction}, например), невозможно перенумеровать спины таким образом, чтобы после преобразования получилась майорановская цепочка, в которой взаимодействуют только соседи. Поэтому приходится менять вид преобразования.

\begin{figure}
    \centering
    \captionsetup{width=0.45\textwidth}
    \subfloat[Нумерация спинов в спиновой системе, необходимая для проведения преобразования Йордана-Вигнера.]{
        \centering%
        \includegraphics[width=0.5\textwidth]{spin_delta}%
        \label{fig:spin_delta}%
    }%
    \subfloat[Спиновая система с центральным спином и его вкладом во взаимодействие первых спинов цепочек.]{
        \centering%
        \includegraphics[width=0.5\textwidth]{spin_delta_central}%
        \label{fig:spin_delta_central}%
    }
    \caption{$\Delta$-образное соединение цепочек.}
\end{figure}

Перенумеруем спины так, как указано на рис. \ref{fig:spin_delta}, то есть рассмотрим три ветви, сходящихся в соединении, и обозначим $n$-ый от соединения спин из ветви $\alpha$ как $\sigma_\alpha (n)$ . Соединение в этой схеме более общего вида, оно называется $\Delta$-образным соединением. При необходимости, оно сводится к Т-образному соединению обнулением одной из констант связи.

Проведём теперь преобразование следующим образом. В каждой из ветвей сделаем своё преобразование Йордана-Вигнера. Тогда в каждой ветви взаимодействовать будут только соседи, а коммутационные соотношения будут как у майорановских фермионов. Однако полученные таким образом операторы из разных ветвей будут коммутировать, а не антикоммутировать. Для установления правильных коммутационных соотношений добавим в преобразование ветви с номером $\alpha$ матрицу $S^\alpha$. Чтобы не изменились коммутаторы внутри отдельных цепочек, квадрат $S^\alpha$ должен быть равен единице. Чтобы майорановские фермионы из разных ветвей антикоммутировали, матрицы $S^\alpha$ должны попарно антикоммутировать. Поскольку ветви три, удобно выбрать в качестве $S^\alpha$ матрицы Паули - они обладают всеми нужными свойствами.

Модифицированное преобразование тогда выглядит следующим образом:
\begin{align}
    \gamma_{2n-1} &= \sigma^z (n) \sum\limits_{p=1}^{n-1} \sigma^x (p), & 
    \gamma_{2n} &= \sigma^y (n) \sum\limits_{p=1}^{n-1} \sigma^y (p).
\end{align}
Оно восстанавливает необходимые коммутационные соотношения и вид гамильтониана внутри ветвей. На соединении, однако, взаимодействие получается не квадратичным и включает в себя матрицы $S^i$.
\begin{equation}
    H_{junc} = - \sum\limits_{\alpha < \beta} J_{\alpha \beta} \sigma_\alpha^z (1) \sigma_\beta^z (1) = - i \sum\limits_{\alpha < \beta} J_{\alpha \beta} \gamma_\alpha (1) \gamma_\beta (1) \epsilon^{\alpha \beta \zeta} S^\zeta.
\end{equation}

Чтобы сделать гамильтониан квадратичным в майорановском представлении, изменим систему так, как показано на рис. \ref{fig:spin_delta_central}, то есть сделаем взаимодействие трёхспиновым в исходной, спиновой системе. Помимо первых спинов цепочек туда теперь будет входить оператор дополнительного спина, который в дальнейшем будет называться ``центральным''. Взаимодействие будет выглядеть следующим образом:
\begin{equation}
    H_{junc} = - \frac{1}{2} \sum\limits_{\alpha \neq \beta \neq \zeta} J_{\alpha \beta} \sigma_\alpha^z (1) \sigma_\beta^z (1) S^\zeta = \frac{i}{2} \sum\limits_{\alpha \beta \zeta} J_{\alpha \beta} \gamma_\alpha (1) \gamma_\beta (1) \epsilon^{\alpha \beta \zeta}.
    \label{eq:junc_H}
\end{equation}

Таким образом, спиновая система с рисунка \ref{fig:spin_delta_central} является спиновым представлением системы майорановских фермионов с квадратичным гамильтонианом и взаимодействием ближайших соседей, в которой удобно проводить переплетение.

Вопросы физической реализации этой системы оставим до части \ref{sec:spin_braiding}, в которой конструкция будет несколько изменена. 

\pagebreak

\subsection{Свойства спиновой системы} \label{subsec:spin_properties}

Рассмотрим теперь, чему соответствуют краевые моды и процессы с ними на спиновом языке. Следующий материал частично описан в \cite{arrays} и \cite{main}.

Будем считать, что поперечное магнитное поле равно нулю внутри топологического участка и бесконечности - вне его. Это удобно с теоретической точки зрения, поскольку основное состояние принимает в таком случае простой вид, который будет указан далее, а с практической просто означает $h \ll J$ и $h \gg J$, что реализуемо. Изменение магнитного поля, приводящее к изменению размеров топологического участка, будем предполагать адиабатическим. Взаимодействие ферромагнитное, $J>0$ в гамильтониане (\ref{eq:spin_H}).

В таком случае спины за пределами топологического участка оказываются эффективно ``замороженными'' вдоль оси $х$. Внутри же появляется изинговская цепочка без магнитного поля с эффективно свободными условиями на границах. Основное состояние такой системы двояко вырождено, базисными векторами можно считать состояния ``все спины вверх'', $\ket{\uparrow \dotsm \uparrow}$, и ``все спины вниз'', $\ket{\downarrow \dotsm \downarrow}$, где в кет-векторах перечислены только спины топологического участка. Если температура достаточно низка, так что возбуждениями можно пренебречь, то все интересующие нас процессы происходят в двумерном подпространстве, построенном на этих векторах. 

Нетрудно показать, как влияет расширение топологического участка на его состояние. Для примера рассмотрим двухспиновый гамильтониан, имея в виду, что второй спин находится внутри топологического участка, а у первого отношение коэффициентов меняется так, что он переходит из топологически тривиальной фазы в топологическую.
\begin{equation}
    H = -J \sigma_1^z \sigma_2^z - h \sigma_1^x.
    \label{eq:32_model_H}
\end{equation}
Легко найти его собственные значения $\lambda = \pm \sqrt{J^2 + h^2}$ и собственные функции, соответствующие основному состоянию,
\begin{gather*}
    \psi_1 = \frac{\left( J + \sqrt{J^2 + h^2} \right) \ket{\uparrow} + h \ket{\downarrow}}{\sqrt{h^2 + \left(J + \sqrt{J^2 + h^2} \right)^2}} \ket{\uparrow}, \\
    \psi_2 = \frac{ h \ket{\uparrow} + \left( J + \sqrt{J^2 + h^2} \right) \ket{\downarrow}}{\sqrt{h^2 + \left(J + \sqrt{J^2 + h^2} \right)^2}} \ket{\downarrow}.
\end{gather*}
В пределе $h \rightarrow \infty$ они становятся состояниями $\ket{\uparrow_x \uparrow_z}$ и $\ket{\uparrow_x \downarrow_z}$, где индекс обозначает, для какой матрицы Паули это состояние является собственным. Когда $h = 0$, они переходят в $\ket{\uparrow_z \uparrow_z}$ и $\ket{\downarrow_z \downarrow_z}$ соответственно.

Из предыдущего следует, что при ``размораживании'' спина, адиабатическом понижении поперечного магнитного поля от $\infty$ до $0$, произойдёт следующий переход:
\begin{equation*}
    \left( \frac{\ket{\uparrow} + \ket{\downarrow}}{\sqrt{2}} \right) (\alpha \ket{\uparrow} + \beta \ket{\downarrow}) \rightarrow \alpha \ket{\uparrow\uparrow} + \beta \ket{\downarrow\downarrow}.
\end{equation*}

Для спиновой цепочки это значит, что состояние топологического участка не меняется при передвижении его границ -  до тех пор, пока граница не проходит через центральный спин с трёхспиновым взаимодействием. Это оправдывает введение следующих обозначений.

Обозначим $\ket{\uparrow}^\tau = \ket{\uparrow \dotsm \uparrow}$, $\ket{\downarrow}^\tau = \ket{\downarrow \dotsm \downarrow}$, где вновь указаны только спины топологического участка. В дальнейшем будем использовать операторы $\tau_x$, $\tau_y$, $\tau_z$, действующие в псевдоспиновом пространстве этих векторов. Здесь и далее топологический участок будет называться просто ``кубитом''. Движение границ в таких обозначениях не отражается на состоянии системы, что удобно. Когда будет необходимо, для обозначения центрального спина будем использовать индекс $s$, а для кубита - $\tau$, например $\ket{\uparrow}^s$, $\ket{\downarrow}^\tau$. 

Осталось понять, как топологический участок проходит через центральный спин. Начнём со случая, когда во взаимодействии участвует компонента $S^z$, а спин находится в состоянии $\ket{\uparrow}^s$ по $z$. Тогда в гамильтониане на соединении (\ref{eq:junc_H}) $S^z$ станет единицей, а само взаимодействие не будет отличаться от всех остальных связей цепочки. Значит, при прохождении через него состояние кубита не изменится.

С другой стороны, если центральный спин находится в состоянии $\ket{\downarrow}^s$, верно всё то же, но взаимодействие меняет знак и становится антиферромагнитным. Легко понять, что в этом случае ``размороженный'' спин окажется перевёрнут относительно соседнего - формула (\ref{eq:transition_states}) иллюстрирует утверждение.

В момент перехода состояние системы является суперпозицией следующих векторов:
\begin{equation}
\begin{split}
    \ket{\dotsm \uparrow \uparrow} \ket{\uparrow}^s \ket{\uparrow \uparrow \dotsm}, \\
    \ket{\dotsm \downarrow \downarrow} \ket{\uparrow}^s \ket{\downarrow \downarrow \dotsm}, \\
    \ket{\dotsm \uparrow \uparrow} \ket{\downarrow}^s \ket{\downarrow \downarrow \dotsm}, \\
    \ket{\dotsm \downarrow \downarrow} \ket{\downarrow}^s \ket{\uparrow \uparrow \dotsm}.
\end{split}
\label{eq:transition_states}
\end{equation}
Видно, что переход действует на кубит как оператор $CNOT$, controlled not, контролируемый оператором $S^\alpha$. Например, в базисе ($\ket{\uparrow\uparrow}$, $\ket{\uparrow\downarrow}$, $\ket{\downarrow\uparrow}$, $\ket{\downarrow\downarrow}$), где первый спин - $S$, а второй - $\tau$,
\begin{equation}
CNOT_{S^z} = 
\begin{pmatrix}
    1 & 0 & 0 & 0 \\
    0 & 1 & 0 & 0 \\
    0 & 0 & 0 & 1 \\
    0 & 0 & 1 & 0
\end{pmatrix},
\end{equation}
причём контролирующий оператор указан в индексе.

\pagebreak

\subsection{Переплетение на спиновом языке} \label{sec:spin_braiding}
На введённом языке удобно рассматривать операцию переплетения, braiding. Будем работать в представлении Гейзенберга и исследовать эволюцию операторов $\vec{\tau}$ и $\vec{S}$. Для этого представим $CNOT$ в виде
\begin{equation}
    CNOT_{S^\alpha} = \frac{S^\alpha + 1}{2} + \frac{S_\alpha - 1}{2} \tau_x 
    = S^\alpha \frac{1 + \tau_x}{2} + \frac{1 - \tau_x}{2},
\end{equation}
и, поскольку $CNOT$ и есть оператор эволюции, нужно вычислить
\begin{equation}
    O' = CNOT_{S^\alpha}^\dagger \cdot O \cdot CNOT_{S^\alpha}.
    \label{eq:cnot_transformation}
\end{equation}
Учитывая коммутационные соотношения, получим вспомогательные равенства
\begin{gather}
    (1 + \tau_x)^2 = 1 + \tau_x^2 + 2 \tau_x = 2 (1 + \tau_x), \\
    (1 + \tau_x) (1 - \tau_x) = 1 - \tau_x^2 = 0, \\
    (1 + \tau_x) \tau^{y,z} (1 + \tau^x) = (1 + \tau_x) (1 - \tau_x) \tau^{y,z} = 0, \\
    (1 + \tau_x) \tau^{y,z} (1 - \tau^x) = (1 + \tau_x)^2 \tau^{y,z} = 2 (1 + \tau_x) \tau^{y,z}.
\end{gather}

Вычисляя результат эволюции с помощью формулы \ref{eq:cnot_transformation} для всех операторов, получим, что при прохождении через соединение со взаимодействием, включающим компоненту спина $S^\alpha$, операторы меняются следующим образом:
\begin{align}
    \tau^{y,z} &\rightarrow S^\alpha \tau^{y,z}, & \tau^x &\rightarrow \tau^x, \\
    S^{!\alpha} &\rightarrow \tau^x S^{!\alpha}, & S^\alpha &\rightarrow S^\alpha,
    \label{eq:spin_transformation}
\end{align}
где $!\alpha$ означает компоненты, отличные от $\alpha$.

Уже известно, как выглядит переплетение на майорановском языке (рис. \ref{fig:braiding}). Проведём те же действия в спиновой системе. При этом топологический участок сначала пройдёт через взаимодействие с $S^z$, затем, на пути вниз на рис. \ref{fig:spin_delta}, через $S^x$, и потом через взаимодействие с $S^y$. $\tau^x$ просто сохраняется всегда; каждый из $S^\alpha$ за время переплетения один раз сохраняется и два раза умножается слева на $\tau^x$, причём $\left( \tau^x \right)^2 = 1$. Так что изменятся только $\tau^y$ и $\tau^z$ составляющие.
\begin{equation}
    \tau^{y,z} \rightarrow S^z \tau^{y,z} \rightarrow \tau^x S^z S^x \tau^{y,z} \rightarrow \left( \tau^x \right)^3 S^z S^x S^y \tau^{y,z} = i \tau^x \tau^{y,z} = \mp \tau^{z,y}.
\end{equation}
Таким образом, переплетение является поворотом на $\pi/2$ вокруг оси $x$ в пространстве псевдоспинового оператора кубита.

Рассмотрев подробно операцию переплетения, можно выделить ключевые моменты и попытаться упростить физическую систему или набор действий, которые нужно с ней совершить. Ниже будет изложено три таких упрощения: отсутствие необходимости проводить кубит через три \textit{различных} соединения; связанная с этим возможность совершать переплетение в одномерной цепочке; и возможность делать два проведение кубита вместо трёх. 

Можно заметить, что все изменения состояния кубита происходят в момент прохода через центральный спин. В отличие от майорановской системы, состояние спиновой меняется не из-за того, что изменилось пространственное положение нулевых мод; вся существенная эволюция связана с проходами через центральный спин. Поэтому использование трёх разных соединений, включающих разные компоненты спина, не является обязательным. Можно использовать одно, поворачивая при этом центральный спин так, чтобы в гамильтониане взаимодействия оказывались нужные компоненты спина. Более того, сложная система с тремя ветвями становится излишней, поскольку при таком проведении операции переплетения кубит никогда не оказывается в третьей ветви.

Эта идея становится ещё важнее в связи с тем, что физическая реализация системы, представленной на рис. \ref{fig:spin_delta_central}, ещё не придумана, и задача не выглядит тривиальной: нужно создать три трёхспиновых взаимодействия, включающие разные компоненты одного и того же спина. С другой стороны, способ реализации одного такого взаимодействия описан в \cite{main}. Чтобы в гамильтониане соединения возникали разные компоненты спина, можно просто поворачивать спин.

Таким образом, значительно удобней была бы система, представленная на рис. \ref{fig:1d_spin_system}. Переплетение в ней поэтапно показано на рис. \ref{fig:1d_braiding}. Сначала происходит проведение кубита через $S^z$; затем спин поворачивают так, что во взаимодействии начинает участвовать $S^x$; кубит проводится обратно в левую часть схемы; спин вращают ещё раз, кубит уводят направо, после чего центральный спин поворачивают ещё раз, возвращая в исходное положение. Последний поворот необходим, чтобы перевести систему в то состояние, в котором она находилась бы после трёх проходов через три различных соединения. Происходит та же самая эволюция, (\ref{eq:spin_transformation}), но в значительно более реализуемой системе.

\begin{figure}
    \centering
    \includegraphics[width=\textwidth]{1d_spin_system}
    \caption{Одномерная спиновая система; компонента спина на соединении контролируется экспериментатором.}
    \label{fig:1d_spin_system}
\end{figure}

\begin{figure}
    \centering
    \captionsetup[subfigure]{justification=centering}
    \subfloat[]{
        \centering%
        \includegraphics[width=0.33\textwidth]{1d_spin_systemA}%
    }%
    \subfloat[]{
        \centering%
        \includegraphics[width=0.33\textwidth]{1d_spin_systemB}%
    }%
    \subfloat[]{
        \centering%
        \includegraphics[width=0.33\textwidth]{1d_spin_systemC}%
    }
    \caption{Переплетение в одномерной системе.}
    \label{fig:1d_braiding}
\end{figure}

Ранее было сказано о том, что если центральный спин находится в состоянии ``вверх'', соответствующем компоненте спина, участвующей во взаимодействии, то проведение кубита через соединение не меняет состояние кубита. Последний проход в данной схеме происходит через компоненту $S^y$; если бы в этот момент спин находился в состоянии ``вверх'' по $y$, то этот проход можно было бы не выполнять. Зная, как изменяются компоненты спина в процессе проходов, (\ref{eq:spin_transformation}), можно убедиться, что к моменту третьего проведения кубита через спин $y$-компонента последнего совпадает с той, что была в начале. Значит, если до начала операции переплетения центральный спин был в состоянии $\ket{\uparrow}_y$, то третий проход можно не осуществлять.

Отсутствие третьего прохода удобно потому, что общее время операции уменьшается в полтора раза. Шумы и малое время когерентности ограничивают возможное количество операций над кубитом, поэтому уменьшение времени выполнения отдельной операции - это положительный результат. Кроме того, в такой схеме кубит после переплетения оказывается в том же месте цепочки, в котором он был в начале, - нет побочного эффекта смещения топологического участка относительно центрального спина.

Таким образом $\Delta$-образное соединение и три проведения кубита через разные компоненты центрального спина сводятся к одномерной цепочке, в которой происходит только два прохода и повороты центрального спина. Как будет указано ниже, совершение трёх проведений кубита через спин имеет определённые преимущества, так что в дальнейшем на количество проходов будут указывать слова ``схема с двумя (тремя) проходами''.

\pagebreak

\section{Отклонения от идеальности в спиновой системе} \label{sec:results}

Ранее предполагалось, что все операции выполняются точно, а шумы, присущие любой физической системе, отсутствуют. В этой части дипломной работы будут приведены качественные и количественные оценки влияния этих неидеальностей на результат переплетения. Предполагая их малыми, вычислим первые порядки всех эффектов и, по возможности, сравним их.

В число этих неидеальностей входят шумы, действующие на спины системы. Их удобно рассматривать отдельно на нескольких этапах. Одним из таких этапов является проход, когда центральный спин находится внутри топологического участка. Другая часть переплетения, когда кубит и спин пространственно разделены, состоит из промежутков времени, когда спин находится в нулевом внешнем поле, и когда спин поворачивают, помещая в магнитное поле на короткий промежуток времени.

Кроме того, будет исследовано влияние неточности проведения операций, таких как получение начального состояния спина в схеме с двумя проходами и поворот спина между проведениями кубита через центральный спин.

\subsection{Обозначения}

Будем предполагать, что до начала переплетения центральный спин и кубит расцеплены, причём кубит обладает какой-либо допустимой матрицей плотности, а спин, в зависимости от количества проходов в схеме, находится либо в произвольном состоянии (три прохода), либо направлен вверх по $y$ (два).
\begin{gather}
    \rho_\tau^{init} = \frac{1}{2} (1 + \vec{r}\vec{\tau}) = \frac{1}{2}
    \begin{pmatrix}
        1 + z & x - i y \\
        x + i y & 1 - z
    \end{pmatrix}, \quad \abs{\vec{r}} \le 1,
    \label{eq:qubit_init}
    \displaybreak[0]\\
    \rho_s^{init} = \frac{1}{2} \begin{pmatrix}
        1 & -i \\
        i & 1
    \end{pmatrix},
    \displaybreak[0]\\
    \rho^{init} = \rho_\tau^{init} \otimes \rho_s^{init}.
\end{gather}

Выполняемые преобразования сводятся к CNOT, причём ``контролирующими'' являются разные компоненты спина $S$. Получить эти матрицы можно как непосредственно рассматривая действие на базисные векторы, так и поворотом $CNOT_{S^z}$.
\begin{gather}
    U_3 = CNOT_{S^z} = \begin{pmatrix}
        1 & 0 & 0 & 0 \\
        0 & 1 & 0 & 0 \\
        0 & 0 & 0 & 1 \\
        0 & 0 & 1 & 0
    \end{pmatrix},
    \displaybreak[0]\\
    U_1 = CNOT_{S^x} = \frac{1}{2} 
    \begin{pmatrix}
        1 & 1 & 1 & -1 \\
        1 & 1 & -1 & 1 \\
        1 & -1 & 1 & 1 \\
        -1 & 1 & 1 & 1
    \end{pmatrix},
    \displaybreak[0]\\
    U_2 = CNOT_{S^x} = \frac{1}{2} 
    \begin{pmatrix}
    1 & 1 & -i & i \\
    1 & 1 & i & i \\
    -i & i & 1 & 1 \\
    i & -i & 1 & 1
    \end{pmatrix}.
\end{gather}

После выполнения переплетения в схеме с двумя проходами
\begin{gather}
    \rho^{final} = U_1 U_3 \rho^{init} (U_1 U_3)^\dagger = \rho^{final}_\tau \otimes \rho^{final}_s,
    \displaybreak[0]\\
    \rho^{final}_\tau = \frac{1}{2}
    \begin{pmatrix}
        1 - y & x - i z \\
        x + i z & 1 + y
    \end{pmatrix} = \frac{1}{2} (1 + x \tau_x + z \tau_y - y \tau_z), 
    \label{eq:tau_final}
    \displaybreak[0]\\
    \rho^{final}_s = \rho^{init}_s.
\end{gather}
К этому результату в следующих параграфах будут найдены различные поправки. Обычно будет получаться так, что кубит и спин из-за неидеальности системы в конце операции останутся зацепленными (скоррелированными); в таком случае будет приводиться в том числе и редуцированная матрица плотности кубита, после взятия следа по состояниям центрального спина. 
Такой подход хорошо характеризует систему, если кубит и данный спин в дальнейшем не будут взаимодействовать, или если центральный спин перед дальнейшими операциями заново инициализируется, устанавливается в определённое положение, и корреляции исчезают.

\subsection{Предположения} \label{subsec:assumptions}

Существуют различные источники ошибок, влияющих на результат операции переплетения. Одним из них являются шумы, влияющие как на спины цепочек, так и на центральный спин. Другой источник - неточности проведения операций над системой.

Как упомянуто в \cite{main}, к центральному спину предъявляются дополнительные технические требования, связанные с особенностями реализации трёхспинового взаимодействия. 
Из-за этого его время корреляции, скорее всего, будет меньше времён корреляции всех остальных спинов цепочки (к первым спинам цепочек никаких дополнительных требований не предъявляется). 
Поэтому в дальнейшем мы будем пренебрегать влиянием шума на спины цепочки, рассматривая в первую очередь результат действия шумов на центральный спин.

Этот процесс, в свою очередь, удобно рассматривать в несколько этапов, которые в первом порядке теории возмущений можно будет сложить. 
Первым таким этапом следует считать всё время до начала операции, в течение которого центральный спин взаимодействует с окружающей средой и переходит в смешанное состояние. 
Поскольку этот этап может идти потенциально неограниченное количество времени, его влияние на результат переплетения может оказать ведущим.

Другим этапом является промежуток времени, в течение которого центральный спин находится внутри топологического участка. 
В схеме с двумя проходами он встретится два раза, в схеме с тремя, соответственно, три (про количество проходов написано в конце \ref{sec:spin_braiding}). 
Кроме того, есть промежутки времени, в течение которых центральный спин находится вне топологического участка. 
В них можно рассмотреть часть, на протяжении которой кубит отводится от центрального спина, и последний не находится во внешнем поле, и время поворота, когда спин помещается во внешнее магнитное поле.

Среди неточностей проведения операций будут рассмотрены ошибки поворота центрального спина и погрешности его инициализации в схеме с двумя проходами.

В работе \cite{flux_qubit} описаны последние достижения в области реализации потоковых кубитов. Указано, что получены времена релаксации $T_1 = 55 \mu s$ и $T_2 \approx 2 T_1$, что означает $T_2^* \gg T_1$. Как нетрудно заметить, именно влияние поперечного шума подавлено в системе, рассматриваемой в данной дипломной работе, поскольку переход $\ket{\uparrow}^\tau \rightleftarrows \ket{\downarrow}^\tau$ появляется лишь в порядке теории возмущений, равном числу узлов в топологическом участке.

\subsection{Начальное состояние центрального спина в схеме с тремя проходами} \label{subsec:spin_init}

Здесь рассмотрим шум, действовавший на центральный спин до начала переплетения. 
Нет никаких ограничений на время действия этого шума, так что следует предполагать, что изначально спин находится в произвольном смешанном состоянии. 
Если бы это оказывало влияние на результат переплетения, эту проблему пришлось бы решать, инициализируя спин, что внесло бы дополнительные неточности и сложность в процесс. 
Однако, как будет показано ниже, начальное состояние центрального спина не влияет на результат в схеме с тремя прогонами.

Из сравнения процесса переплетения в майорановской и спиновой системах заранее очевидно, что в схеме с тремя проходами центральный спин может находиться в произвольном \textit{чистом} состоянии, это не должно повлиять на результат операции.
Интересно, что даже если спин изначально находился в смешанном состоянии, то есть был зацеплен с внешней средой, кубит не переходит в смешанное состояние в результате переплетения. 
Здесь имеются в виду только различные начальные состояния спина; шумы, как и неточности оборудования, предполагаются отсутствующими.

Это можно проверить непосредственным вычислением, посчитав, что
\begin{equation}
    U_2 U_1 U_3 (\rho_\tau \otimes \rho_s) (U_2 U_1 U_3)^\dagger = \rho^{final}_\tau \otimes \rho_s
    \label{eq:u213_general},
\end{equation}
где $\rho^{final}_\tau$ из ($\ref{eq:tau_final}$), а $\rho_s$ произвольная допустимая.

Впрочем, это можно понять и так. В гейзенберговском представлении матрица плотности вообще не меняется; законы преобразования операторов не зависят от того, является состояние чистым или смешанным; а для чистых состояний, как было указано выше, результат очевиден из сравнения с майорановской системой.

Этот результат имеет качественное значение: несмотря на недостатки схемы с тремя проходами, нет необходимости делать хоть какие-то операции со спином до начала переплетения (инициализировать, например), это не повлияет на результат. 
В схеме с двумя проходами спин должен находиться в особом начальном состоянии, что служит дополнительным источником ошибок - невозможно идеально задать состояние спина.

\subsection{Спин внутри топологического участка и поперечный шум} \label{subsec:inside_transverse}

Вспомним общее разделение на этапы, \ref{subsec:assumptions}. 
Из предположений следует, что центральный спин подвержен действию шума всё время переплетения, и это действие должно передаваться взаимодействующему с ним кубиту. Оказывается, однако, что на одном из этапов процесса часть шума, действующего на центральный спин, подавлена.

Назовём ``поперечным'' и ``продольным'' шумом те составляющие шума, параллельные и перпендикулярные компоненте спина, участвующей в трёхспиновом взаимодействии.
Пока спин находится внутри топологического участка, низкая температура и длина топологического участка подавляют продольную релаксацию (соответствующую поперечному шуму).

Внимательный взгляд на волновые функции (\ref{eq:transition_states}) основного состояния системы в таком положении показывает, что в этом основном состоянии нет пар функций, получающихся друг из друга просто переворотом центрального спина: есть только пары, получающиеся переворотом его и одной из частей топологического участка, слева или справа от самого спина.

Это означает, что переходы, происходящие под влиянием поперечного шума, появляются только в $(1 + \min(m,n))$-ом порядке теории возмущений, где $m$ - число спинов топологического участка слева от спина, а $n$ - справа. 
При этом возмущения должны действовать на центральный спин и все спины топологического отрезка с одной из сторон от спина.
Очевидно, что пока спин находится внутри топологического участка, порядок теории возмущений - не первый.

\subsection{Cпин внутри топологического участка и продольный шум} \label{subsec:inside_longitudinal}

Выше, в \ref{subsec:inside_transverse}, было показано, что влиянием поперечного шума на центральный спин внутри топологического участка можно пренебречь. Однако в системе нет никакой защиты от продольного шума. Вычислению его влияния на результат переплетения посвящён этот параграф.

В процессе переплетения спин находится внутри топологического участка два раза: когда во взаимодействии стоит $S^z$ и $S^x$.
За это время между состояниями $\ket{\uparrow}_\alpha$ и $\ket{\downarrow}_\alpha$, где $\alpha = z, x$, набегает случайная фаза.
Следует обратить внимание на то, что во втором случае, когда речь идёт об $S^x$, шум остаётся продольным, но уже вдоль оси $x$. 
Это связано с тем, что взаимодействие имеет вид $\tau^z S^x \tau^z$, и от поперечного $x$ шума система защищена так, как описано ранее. 
Тот же аргумент становится очевидным в шрёдингеровском представлении: если в гейзенберговском менялся оператор спина, участвующий во взаимодействии, то в шрёдингеровском взаимодействие не меняется вообще, поворачивается лишь спин.
Поэтому не является удивительным то, что поправка, набирающаяся за эти проходы, одинакова.

Представляя $CNOT$ в виде
\begin{equation}
U_3 = CNOT_{S^z} = 
\begin{pmatrix}
e^{i \phi} & 0 & 0 & 0 \\
0 & e^{i \phi} & 0 & 0 \\
0 & 0 & 0 & e^{-i \phi} \\
0 & 0 & e^{-i \phi} & 0
\end{pmatrix},
\end{equation}
вычисляя $\rho^{final}$, усредняя его по реализациям шума и вычисляя частичный след по состояниям центрального спина, получаем
\begin{equation}
\rho^{final}_\tau = \frac{1}{2}
\begin{pmatrix}
1 - y e^{-t/T_2} & x - i z e^{-t/T_2} \\
x + i z e^{-t/T_2} & 1 + y e^{-t/T_2}
\end{pmatrix}
\end{equation}
где $t$ - это время прохода, $T_2$ - время поперечной релаксации. 

\subsection{Неточности проведения операций} \label{subsec:operations}

Пусть начальное состояние спина немного отличается от $\ket{\uparrow}_y$. Рассмотрим влияние этого отличия на результат в первом порядке. 
Пусть начальная волновая функция центрального спина
\begin{equation}
    \psi = \ket{\uparrow} + \epsilon \ket{\downarrow},
\end{equation}
причём поправка к волновой функции из-за нормировки второго порядка, а соответствующее изменение матрицы плотности
\begin{equation}
    \delta\rho^{init}_s = \frac{\epsilon}{2}
    \begin{pmatrix}
        1 & 0 \\
        0 & -1 
    \end{pmatrix}.
\end{equation}

Вычисляя $U_1 U_3 \rho^{init} (U_1 U_3)^\dagger$, найдём поправку к результату
\begin{equation}
    \delta \rho^{final} = \frac{\epsilon}{4}
    \begin{pmatrix}
        x & 1 & z & i y \\
        1 & x & - i y & -z \\
        z & i y & -x & -1 \\
        - i y & - z & -1 & -x
    \end{pmatrix}
\end{equation}
и её частичный след по состояниям центрального спина
\begin{equation}
    \delta \rho^{final}_\tau = 
    \begin{pmatrix}
        0 & 0 \\
        0 & 0
    \end{pmatrix}.
\end{equation}

Несмотря на то, что первый порядок поправки к результату более не позволяет разбивать всю матрицу плотности на тензорное произведение матриц плотностей кубита и спина, после усреднения по спиновым состояниям поправка оказывается тождественно равной нулю. Она возникнет только во втором порядке.

Также ошибка может возникнуть во время поворота спина между проходами и после них. Поворот происходит вокруг оси $y$ на $\pi/2$. Пусть угол оказался смещён на $\delta \phi$. Рассматривая $U_1$ как
\begin{gather}
    U_1 = R^\dagger U_3 R,
    \displaybreak[0]\\
    R = \frac{1}{\sqrt{2}}
    \begin{pmatrix}
        1 & 0 & 1 & 0 \\
        0 & 1 & 0 & 1 \\
        -1 & 0 & 1 & 0 \\
        0 & -1 & 0 & 1
    \end{pmatrix},
    \displaybreak[0]\\
    \delta R = \frac{\delta \phi}{\sqrt{2}}
    \begin{pmatrix}
        -1 & 0 & 1 & 0 \\
        0 & -1 & 0 & 1 \\
        -1 & 0 & -1 & 0 \\
        0 & -1 & 0 & -1
    \end{pmatrix},
\end{gather}
где $\delta R$, поправка к матрице поворота, получена разложением в ряд Тейлора по $\delta \phi$.

Соответственно, для изменения результата получается
\begin{multline}
    \delta \rho^{final} = \left( \delta R^\dagger U_3 R U_3 + R^\dagger U_3 \delta R U_3 \right) \rho^{init} (U_1 U_3)^\dagger \\
    + U_1 U_3 \rho^{init} \left( \delta R^\dagger U_3 R U_3 + R^\dagger U_3 \delta R U_3 \right)^\dagger.
\end{multline}
Выражение, в котором уже взят частичный след по спиновым состояниям, выглядит так:
\begin{equation}
    \delta \rho^{final} = \delta \phi
    \begin{pmatrix}
        -z & + i y \\
        - i y & z
    \end{pmatrix},
\end{equation}
где $y,\ z$ - соответствующие компоненты начального состояния кубита, см. (\ref{eq:qubit_init}).
Оба порядка матрицы плотности выглядят следующим образом
\begin{equation}
    \rho^{final}_\tau = \frac{1}{2}
    \begin{pmatrix}
        1 - y -  z \delta \phi & x - i z + i y \delta \phi \\
        x + i z - i y \delta \phi & 1 + y +  z \delta \phi
    \end{pmatrix}.
\end{equation}
Можно заметить, что диагональные элементы матрицы плотности могут стать больше единицы или меньше нуля на $z \ll \delta \phi \ll 1$. Это связано с некорректностью разложения в этом пределе, поскольку величина $z \delta \phi$ становится в этом случае порядка $\delta \phi^2$ или меньше, и с выбранной точностью ею следует пренебречь. Точный ответ для матрицы плотности,
\begin{equation}
    \rho^{final}_\tau = \frac{1}{2}
    \begin{pmatrix}
    1 - y \cos \delta \phi - z \sin \delta \phi & x - i z \cos \delta \phi + i y \sin \delta \phi \\
    x + i z \cos \delta \phi - i y \sin \delta \phi & 1 + y \cos \delta \phi + z \sin \delta \phi
    \end{pmatrix},
\end{equation}
показывает, что появляющийся из $\cos \delta \phi$ квадратичный член станет ведущим на малых $z$. Нетрудно также проверить, что точная матрица плотности, как и следует ожидать, является матрицей плотности при любых значениях параметров.

\pagebreak

\section{Заключение}

В данной дипломной работе было исследовано влияние шумов и неточностей проведения операций на результат переплетения в спиновой системе, которая на майорановском языке соответствует цепочкам майорановских фермионов с взаимодействием ближайших соседей и $Delta$-образным соединением.

В частности, было показано, что в схеме с тремя проходами начальное состояние центрального спина не влияет на результат переплетения (\ref{subsec:spin_init}; о количестве проходов см. конец \ref{sec:spin_braiding}). Этот результат имеет принципиальное значение, поскольку до начала операций центральный спин может релаксировать произвольное количество времени. Шум поэтому входит не как малый параметр; наоборот, матрицу плотности центрального спина следует предполагать произвольной допустимой. Если бы результат переплетения зависел от начального состояния спина, эта зависимость не была бы малой поправкой.
 
Далее, была рассмотрена дополнительная и свойственная только данной системе защита центрального спина от части шумов (\ref{subsec:inside_transverse}). Влияние продольного шума оказывается подавлено, пока центральный спин находится внутри топологического участка (об определении продольного шума в этом контексте см. параграф \ref{subsec:inside_transverse}). Поскольку действие шумов является основным затруднением при построении полнофункциональных квантовых компьютеров, защита даже от части из них, предоставленная самой системой - это важное наблюдение.

Кроме того, было вычислено влияние шумов, действующих на центральный спин, на результат переплетения (\ref{subsec:inside_longitudinal}), как и влияние неточности проведения операций - инициализации центрального спина в схеме с двумя проходами и поворотов его между и после них (\ref{subsec:operations}). Следует отметить, что влияние погрешности инициализации спина в первом порядке теории возмущений равно нулю.

Также был рассмотрен модельный гамильтониан Китаева, в котором впервые были исследованы майорановские нулевые моды (\ref{subsec:kitaev}), как и вопросы реализации переплетения - операции над майорановскими модами, не нарушающей их топологическую защиту (\ref{subsec:braiding}). Далее было кратко изложено содержание \cite{main}: проблемы преобразования системы майорановских фермионов на спиновый язык (\ref{subsec:jordan}), свойства возникающей спиновой системы (\ref{subsec:spin_properties}) и особенности переплетения на этом языке (\ref{sec:spin_braiding}).

Эта работа может быть использована в планируемом эксперименте по реализации рассмотренной спиновой системы на основе сверхпроводящих потоковых кубитов и проведению в ней переплетения.

\pagebreak

\section{Приложения}

\subsection{Шумы в спиновых системах}
Рассмотрим один спин. Шумом для него назовём случайные магнитные поля, наведённые окружающей средой. Если спин находится в сильном (относительно случайных полей) внешнем магнитном поле, то принято выделять поперечную и продольную компоненту шума - перпендикулярную и параллельную полю соответственно. В этом случае спин подчиняется феноменологическим уравнениям Блоха \cite{abragam}
\begin{equation}
\begin{split}
    \dv{s_x(t)}{t} &= \gamma \left( \vec{s}(t) \times \vec{h}(t) \right)_x - \frac{s_x(t)}{T_2}, \\
    \dv{s_y(t)}{t} &= \gamma \left( \vec{s}(t) \times \vec{h}(t) \right)_y - \frac{s_y(t)}{T_2}, \\
    \dv{s_z(t)}{t} &= \gamma \left( \vec{s}(t) \times \vec{h}(t) \right)_z - \frac{s_z(t) - s_0}{T_1}.
\end{split}
\end{equation}
Здесь введены стационарное значение вектора спина $s_0$, а также времена продольной и поперечной релаксации $T_1$ и $T_2$ соответственно. Вектор спина вводится согласно представлению матрицы плотности спина в виде
\begin{equation}
    \rho = \frac{1}{2} (1 + \vec{s} \, \vec{\sigma}).
\end{equation}
При этом для допустимых матриц плотности $\abs{s} < 1$, то есть возможные значения вектора спина образуют шар в соответствующем векторном пространстве. Его поверхность, сфера $\abs{s} = 1$, отвечает чистым состояниям.

Рассматривая сначала только продольный шум, то есть слагаемое в гамильтониане вида $h_z(t) s^z$, приходим к выводу, что он добавляет случайную фазу между состояниями $\ket{\uparrow_z}$ и $\ket{\downarrow_z}$. Усреднение матрицы плотности по реализациям возмущения показывает, что в этом случае недиагональные её элементы затухают экспоненциально со временем когерентности, которое будет обозначено как $T_2^*$ и названо временем поперечной релаксации. Здесь стоит подчеркнуть, что \textit{продольный} шум приводит к \textit{поперечной} релаксации. Везде в тексте явно указано, про поперечность или продольность чего именно идёт речь.

$T_1$, соответственно, названа скорость затухания диагональных элементов матрицы плотности. Можно показать, что при наличии всех видов шумов полное время поперечной релаксации вычисляется согласно
\begin{equation}
    \frac{1}{T_2} = \frac{1}{2 T_1} + \frac{1}{T_2^*}.
\end{equation}
Из этого следует, в частности, что если $T_2 \approx 2 T_1$, то $T_2^* \gg 2 T_1$.

\pagebreak

\bibliographystyle{unsrt}
\bibliography{diploma}

\end{document}