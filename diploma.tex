\documentclass[a4paper,12pt]{article}

%%	Fonts
\usepackage{cmap}
\usepackage{mathtext}
\usepackage{hyperref}
\usepackage[T2A]{fontenc}
\usepackage[utf8]{inputenc}
\usepackage[english,russian]{babel}
\usepackage[scaled]{helvet}
%\usepackage{fullpage}

% 	Math packages
\usepackage{amsmath,amsfonts,amssymb,amsthm,mathtools}
\usepackage{icomma}
\usepackage{euscript}
\usepackage{mathrsfs}
\usepackage{tensor}
\usepackage{physics}
\usepackage{stackrel}
% 	My math operartors
%\mathtoolsset{showonlyrefs=true}
\newcommand{\bigo}[1]   {\,\ensuremath{\mathop{}\mathopen{}\mathcal{O}\mathopen{}\displaystyle\left(#1\right)}}
\newcommand{\smallo}[1]	{\,\scriptstyle\mathop{}\mathopen{}\mathcal{O}\mathopen{}\displaystyle\left(#1\right) }
\DeclareMathOperator{\arcsinh}{arcsinh}
\DeclareMathOperator{\arctanh}{arctanh}
\newcommand*{\hm}[1]{#1\nobreak\discretionary{} {\hbox{$\mathsurround=0pt #1$}}{}}

%%	Graphics
\usepackage{graphicx}
\graphicspath{{images/}}
\setlength\fboxsep{3pt}
\setlength\fboxrule{1pt}
\usepackage{wrapfig}
% 	Tables
\usepackage{array,tabularx,tabulary,booktabs}
\usepackage{longtable}
\usepackage{multirow}
\usepackage{caption}
%\usepackage{subcaption}
\usepackage{subfig}
%	Itemize
\usepackage{enumitem}
\setlist[itemize]{topsep=0pt, leftmargin=0.2in}

%% 	Theorems
\theoremstyle{plain} % Это стиль по умолчанию, его можно не переопределять.
 \newtheorem{theorem}{Теорема}[section]
 \newtheorem{proposition}{Утверждение}
%\newtheorem{proposition}[theorem]{Утверждение}
 \newtheorem{lemma}{Лемма}
\theoremstyle{definition} % "Определение"
 \newtheorem{definition}{Определение}[section]
 \newtheorem{corollary}{Следствие}[theorem]
 \newtheorem{problem}{Задача}[section]
\theoremstyle{remark} % "Примечание"
 \newtheorem*{solution}{Решение}
\renewcommand\qedsymbol{$\blacksquare$}
\newcommand{\proofbegin}{\ensuremath{\blacktriangle}\nopunct}
\newcommand{\contbegin}{\footnotesize{\textcircled{\scriptsize!}}\normalsize \ }
\newcommand{\contend}{\ensuremath{\otimes}}
\newcommand{\vect}[1]{\ensuremath{\overrightarrow{#1}}}

%%	Programming
\usepackage{etoolbox} % логические операторы

%%	Page
\usepackage[14pt]{extsizes} % Возможность сделать 14-й шрифт
\usepackage{geometry} % Простой способ задавать поля
	\geometry{top=20mm}
	\geometry{bottom=20mm}
	\geometry{left=22mm}
	\geometry{right=22mm}
	\geometry{bindingoffset=0mm}
%\linespread{0.01}
%\usepackage{mathpazo}
%usepackage{fancyhdr} % Колонтитулы
% 	\pagestyle{fancy}
% 	\renewcommand{\headrulewidth}{.1mm}  % Толщина линейки, отчеркивающей верхний колонтитул
% 	\lfoot{}
% 	\rfoot{}
% 	\rhead{}
% 	\chead{}
% 	\lhead{}
 	% \cfoot{Нижний в центре} % По умолчанию здесь номер страницы
\usepackage{setspace} % Интерлиньяж
\onehalfspacing % Интерлиньяж 1.5
%\doublespacing % Интерлиньяж 2
%\singlespacing % Интерлиньяж 1
%\linespread{0.01}
\usepackage{lastpage} % Узнать, сколько всего страниц в документе.
\usepackage{soul} % Модификаторы начертания
%\usepackage{hyperref}
\usepackage[usenames,dvipsnames,svgnames,table,rgb]{xcolor}
\hypersetup{					% Гиперссылки
    unicode=true,           	% русские буквы в раздела PDF
    pdftitle={},   				% Заголовок
    pdfauthor={Nestyuk Arseniy},	% Автор
    pdfsubject={},      		% Тема
    pdfcreator={Nestyuk Arseniy}, 	% Создатель
    pdfproducer={}, 			% Производитель
    pdfkeywords={Qubit} {Quantum physics} {Majorana}, % Ключевые слова
    colorlinks=true,       		% false: ссылки в рамках; true: цветные ссылки
    linkcolor=red,          	% внутренние ссылки
    citecolor=green,        	% на библиографию
    filecolor=magenta,      	% на файлы
    urlcolor=cyan           	% на URL
}
\usepackage{multicol} % Несколько колонок

% 	Titel
\author{}
\title{}
\date{\today}

\graphicspath{ {images/} }


\begin{document}
    
\begin{titlepage}
    \begin{center}
        Министерство образования и науки Российской Федерации
        \linebreak
        
        Федеральное государственное автономное
        
        образовательное учреждение высшего образования 
        
        «Московский физико-технический институт 
        
        (государственный университет)»
        \linebreak
        
        Факультет общей и прикладной физики
        
        Кафедра проблем теоретической физики
        \vspace{100pt}
        
        \textbf{\LARGE Майорановские кубиты в джозефсоновских изинговских цепочках}
        \linebreak
        
        Выпускная квалификационная работа бакалавра
     \end{center}
     \vspace{90pt}
     
     \begin{flushright}
         \begin{tabular}{cl}
         &Выполнил:\\
         &студент 322 группы\\         
         &Нестюк А.П.\\
         &\\
         &Научный руководитель:\\
         &д.ф-м.н., проф. Махлин Ю.Г.
        \end{tabular}
    \end{flushright}
    \vspace{60pt}
    
    \begin{center}
        Черноголовка, 2017
    \end{center}
 \end{titlepage}   

\tableofcontents
\pagebreak

\section{Введение}
В последнее время большой интерес вызывают возможности реализации кубитов на основе так называемых майорановских фермионов. Реализация таких кубитов и системы управления ими может оказаться ключевым фактором в построении квантового компьютера, способного решать реальные задачи.

Впервые майорановские фермионы были рассмотрены Китаевым в \cite{kitaev}. Он рассмотрел фермионную цепочку и провёл над ней преобразование Майораны, показав, что при выполнении определённых соотношений между коэффициентами гамильтониана появляется возбуждение с нулевой энергией, эффективно составленное из фермионных операторов, находящихся на разных концах цепочки. Физические, локальные шумы меняют энергию этого возбуждения экспоненциально слабо по длине цепочки. Потеря когерентности из-за влияния шумов является одной из основных проблем создания кубитов, и данная система практически полностью решает этот вопрос.

К сожалению, решая один вопрос, майорановские фермионы ставят новые. Во-первых, их техническая реализация ещё далека от совершенства. Самой многообещающей основой для создания таких систем сейчас считаются p-волновые сверхпроводники {\color{red}ссылка}, и в них получены только признаки существования таких фермионов. Об управлении ими речь пока не идёт.

Во-вторых, строго доказано, что нельзя выполнить произвольный квантовый алгоритм, не нарушив защиту майорановских фермионов. В науке о квантовых алгоритмах есть понятие "универсального набора операций"$ $. Если такой набор операций реализуем на данной физической системе, то возможно выполнение произвольного действия над данной системой, и тем выполним любой алгоритм, в противном случае это невозможно. Над майорановскими фермионами реализовать универсальный набор операций не получится.

Тем не менее, интерес к майорановским фермионам не угас. Технические сложности обычно преодолимы, и даже если не получится производить над такими кубитами нужные операции, их можно будет использовать в качестве долговременной памяти. 

Более того, оказались простимулированы исследования, связанные с прочими системами, которые можно описывать на языке майорановских фермионов. Одной из таких систем является спиновая цепочка{\color{red}, над которой совершено преобразование} Йордана-Вигнера. Очевидно, получившуюся фермионную систему можно свести к системе майорановских фермионов. Однако, поскольку преобразования Йордана-Вигнера нелокально, обычные локализованные шумы спиновой системы становятся нелокальными шумами в фермионной системе, из-за чего теряется защита от шумов - основная причина интереса. Тем не менее, поскольку без разрушения защиты невозможно выполнять квантовые алгоритмы, можно нарушать её и в такой форме. Поскольку спиновые системы на основе джозефсоновских кубитов изучены относительно хорошо, а времена их когерентности позволяют проводить с ними эксперименты, можно уже сейчас проверить, как соотносится теория с практикой. Кроме того, возможно, что майорановские кубиты на сверхпроводниках будут использоваться в основном как память компьютера, а спиновые системы, полностью находящиеся под нашим контролем, - как способ проводить операции и хранить данные небольшое количество времени (процессор в терминах современных компьютеров).

В работе \cite{main} описаны теоретические основы и элементы практической реализации подобных спиновых систем. Единственная доступная операция над майорановскими фермионами требует выхода за пределы одномерной цепочки, что представляет дополнительные сложности в момент выполнения преобразования Йордана-Вигнера. Дальнейшие возникающие затруднения и указанные в статье методы их устранения будут приведены ниже.

Целью данной работы является более подробное изучение спиновой системы, описанной в статье \cite{main}. Будут рассмотрены различные погрешности, включая влияние различных шумов на состояние кубита и фактический размер майорановских фермионов. Будут сделаны некоторые общие выводы относительно защищённости этой системы от случайных шумов.

{\color{red}А здесь можно привести содержание описания (в главе х рассказывается про у...)}

% Moving to next section
% Majorana modes in fermionic systems

\pagebreak

\section{Майорановские моды в фермионных системах}

Основой для всего дальнейшего служат статьи \cite{kitaev}, в которой вводится понятие майорановских фермионов, и \cite{braiding}, где предлагается методика выполнения операции над ними.

\subsection{Модель Китаева}
Следуя \cite{kitaev}, рассмотрим фермионную цепочку со следующим гамильтонианом:
\begin{equation}
    H = \sum\limits_j -\omega (a_j^\dagger a_{j+1} + a_{j+1}^\dagger a_j - a_j a_{j+1} - a_j^\dagger a_{j+1}^\dagger) - \mu \left( a_j^\dagger a_j - \frac{1}{2} \right)
\end{equation}
(строго говоря, в гамильтониане Китаева коэффициенты перед $a_j^\dagger a_{j+1}$ и $a_j a_{j+1}$ разные, но для простоты здесь они положены одинаковыми).

Майорановскими операторами называется набор операторов
\begin{align}
\gamma_{2j-1} &= a_j + a_j^\dagger, & \gamma_{2j} &= i\left( a_j^\dagger - a_j \right),
\end{align}
удовлетворяющий соотношениям
\begin{align}
\gamma_i^2 &= 1, & \{\gamma_i, \gamma_j\} = 2 \delta_{ij},
\end{align}
причём чётные и нечётные индексы более неразличимы.

Гамильтониан после преобразования имеет вид
\begin{equation}
H = -i\mu/2 \sum\limits_{j=1}^N \gamma_{2j-1} \gamma_{2j} + i \omega \sum\limits_{j=1}^{N-1} \gamma_{2j} \gamma_{2j+1}.
\label{eq:main_hamiltonian}
\end{equation}
Нетрудно заметить, что в предельных случаях, когда либо $\mu$, либо $\omega = 0$, получаются два принципиального разных гамильтониана,
\begin{align}
H &= -i\mu/2 \sum\limits_{j=1}^N \gamma_{2j-1} \gamma_{2j}, & H &= i \omega \sum\limits_{j=1}^{N-1} \gamma_{2j} \gamma_{2j+1}.
\end{align}
Во втором отсутствуют члены $\gamma_1$ и $\gamma_{2L}$, которые вместе составляют одно возбуждение с нулевой энергией. Предполагая сохранение чётности, учитывая то, что квадрат майорановского оператора равен 1, и рассматривая первый порядок теории возмущений, получим, что шум, расщепляющий этот уровень, должен быть произведением операторов $a_1$, $a_L$ и их сопряжённых. Такой шум, очевидно, нелокален, и потому нефизичен. Отсюда делается вывод о так называемой топологической защите этого уровня от шума. Слово "топологический"$ $ связано с тем, что майорановские моды локализованы на краях цепочки.

В первом гамильтониане возбуждения с нулевой энергией нет. Как показывает точный расчёт, когда оба коэффициента в гамильтониане отличны от нуля, наличие такого возбуждения зависит от отношения коэффициентов $\mu$ и $\omega$: когда $\mu/\omega > 2$, его нет, в противном случае есть. Причём когда возбуждение есть, оно затухает вглубь образца с экспоненциальной скоростью.

Видно, что образуются две фазы, причём в одной из них на границе "живут"$ $ майорановские фермионы, также называемые краевыми модами. Такую фазу будет далее называть топологической.


\subsection{Переплетение майорановских фермионов}

Поскольку майораноские моды не входят в гамильтониан, возникает проблема управления ими, а без контролируемого изменения состояния кубита невозможны квантовые вычисления.

В \cite{braiding} предлагается операция, названная braiding. Это слово в данной дипломной работе будет переводиться как "переплетение"$ $. Идея состоит в том, что если поменять местами майорановские фермионы, то состояние системы изменится. Это легко понять, введя оператор чётности
\begin{equation}
T = \prod\limits_{j=1}^{2L} (-i \gamma_j).
\end{equation}
Этот оператор коммутирует с гамильтонианом \ref{eq:main_hamiltonian}. В самом деле, в гамильтониан входят только чётные по числу майорановских фермионов слагаемые, а в оператор чётности - каждый из таких фермионов один раз. Учитывая, что различные фермионы антикоммутируют, а сами с собой - коммутируют, получим
\begin{align}
\gamma_n T &= - T \gamma_n & \gamma_n \gamma_m T = T \gamma_n \gamma_m.
\end{align}

Если теперь поменять местами два фермионных оператора, оператор чётности изменит знак:
\begin{multline}
T_{nm} = (-i)^{2N} \gamma_1 \dotsm \gamma_n \dotsm \gamma_m \dotsm \gamma_{2N} = 
    (-i)^{2N} (-1)^{m-n-1} \gamma_1 \dotsm \gamma_n \gamma_m \dotsm \gamma_{2N} \\ =
    (-i)^{2N} (-1)^{m-n} \gamma_1 \dotsm \gamma_m \gamma_n \dotsm \gamma_{2N} = 
     - (-i)^{2N} \gamma_1 \dotsm \gamma_m \dotsm \gamma_n \dotsm \gamma_{2N} = 
     T_{mn} = - T_{nm}.
\end{multline}
Если совершается физическое преобразование, которое в реальном пространстве меняет местами майорановские моды и сохраняет при этом чётность, то операторы (в гейзенберговском представлении) не могут просто переходить друг в друга: один из них меняет знак, чтобы сохранить чётность. Соответственно этому изменится и состояние системы.

В одномерной цепочке невозможно переставить фермионы, не проведя их друг через друга. В \cite{braiding} предлагается система, в которой это возможно. Исследуем изменения относительно предыдущего пункта поэтапно.

Сначала нужно рассмотреть фермионную цепочку, отношение коэффициентов гамильтониана которой показано на рис. \ref{fig:two_phases}. Внутри каждого из участков коэффициенты гамильтониана постоянны, при этом в среднем участке отношение коэффициентов такого, что он находится в топологической фазе, а другие участки - в фазе без фермионов.

\begin{figure}
    \centering
    \begin{minipage}{.5\textwidth}
        \centering
        \includegraphics[width=\linewidth]{two_phases}
        \captionsetup{width=0.9\textwidth}
        \captionof{figure}{Зависимость отношения коэффициентов гамильтониана от номера узла; горизонтальная линия показывает границу фаз.}
        \label{fig:two_phases}
    \end{minipage}%
    \begin{minipage}{.5\textwidth}
        \centering
        \includegraphics[width=\linewidth]{T_junction}
        \captionsetup{width=0.9\textwidth}
        \captionof{figure}{T-соединение; точками показаны фермионы, линиями - их взаимодействие.}
        \label{fig:t_junction}
    \end{minipage}
\end{figure}

В такой системе майорановские фермионы также будут присутствовать; при конечном отношении коэффициентов возбуждение будет экспоненциально затухать не только внутрь топологической фазы, но и "наружу"$ $, в обычную фазу.

Обладая контролем над коэффициентами гамильтониана, можно изменять их отношение так, что граница фаз будет смещаться. При адиабатическом выполнении данной операции состояние майорановских фермионов не будет меняться. Это означает, что мы можем двигать топологический участок в произвольное место цепочки.

Наконец, можно представить себе соединение двух фермионных цепочек. Простейший вариант, представленный на рис. \ref{fig:t_junction} и называемый T-соединением, является обычной фермионной цепочкой, один из узлов которой взаимодействует с крайним фермионом другой цепочки. В такой системе тоже существуют две фазы по отношению коэффициентов гамильтониана, и на границе между ними находятся майорановские фермионы. Соответственно, границы между фазами в такой конструкции можно двигать в том числе и в другие участки соединения.

\begin{figure}
    \centering
    \subfloat{%
        \centering
        \includegraphics[width=0.25\textwidth]{braidingA}}%
    \subfloat{%
        \centering
        \vspace{10pt}
        \includegraphics[width=0.25\textwidth]{braidingB}}%
    \subfloat{%
        \centering
        \includegraphics[width=0.25\textwidth]{braidingC}}%
    \subfloat{%
        \centering
        \includegraphics[width=0.25\textwidth]{braidingD}}%
    \caption{Braiding, переплетение майорановских мод в Т-соединённых фермионных цепочках, \cite{braiding}.}
    \label{fig:braiding}
\end{figure}

Теперь можно понять, как происходит перестановка майорановских мод. Процесс представлен на рис. \ref{fig:braiding}. В вышеописанной системе одна из мод уводится в нижнее ответвление, другая переводится на исходное место первой, после чего первая передвигается туда, где в начале была вторая. Все процессы осуществляются адиабатически и с сохранением чётности.

% Moving to next section
% Spin system

\pagebreak

\section{Спиновая система}

В данной части будет кратко рассмотрено содержание \cite{main}. Будет показано, как можно совершить переход от спиновой системы к фермионной, с учётом того факта, что цепочка не является одномерной и обычное преобразование Йордана-Вигнера не даёт гамильтониана с соседним взаимодействием. Далее будет рассмотрено переплетение на спиновом языке.

\subsection{Модифицированное преобразование Йордана-Вигнера}

Обычное преобразование Йордана-Вигнера (здесь переводящее спиновую систему на язык майорановских фермионов),
\begin{align}
\gamma_{2i-1} &= \sigma^z (n) \sum\limits_{p=1}^{n-1} \sigma^x (p), & 
\gamma_{2i} &= \sigma^y (n) \sum\limits_{p=1}^{n-1} \sigma^y (p),
\end{align}
превращает изинговскую цепочку с поперечным магнитным полем
\begin{equation}
H = - J \sum\limits_{n=1}^{N-1} \sigma^z(n) \sigma^z(n+1) - \sum\limits_{n=1}^N h(n) \sigma^x(n)
\label{eq:spin_H}
\end{equation}
в майорановскую цепочку с квадратичным гамильтонианом и соседним взаимодействием
\begin{equation}
H = -i \sum\limits_{n=1}^N h(n) \gamma(2n-1) \gamma(2n) + i J \sum\limits_{n=1}^{N-1} \gamma(2n) \gamma(2n+1).
\end{equation}

Для этого преобразования важно уметь последовательно перенумеровывать спины. В случае, когда есть три ветки (рис. \ref{fig:t_junction}, например), невозможно перенумеровать спины таким образом, чтобы после преобразования получилась майорановская цепочка, в которой взаимодействуют только соседи. Поэтому приходится менять вид преобразования.

\begin{figure}
    \centering
    \captionsetup{width=0.45\textwidth}
    \subfloat[Нумерация спинов в спиновой системе, необходимая для проведения преобразования Йордана-Вигнера.]{
        \centering%
        \includegraphics[width=0.5\textwidth]{spin_delta}%
        \label{fig:spin_delta}%
    }%
    \subfloat[Спиновая система с центральным спином и его вкладом во взаимодействие первых спинов цепочек.]{
        \centering%
        \includegraphics[width=0.5\textwidth]{spin_delta_central}%
        \label{fig:spin_delta_central}%
    }
    \caption{$\Delta$-соединение цепочек.}
\end{figure}

Перенумеруем спины так, как указано на рис. \ref{fig:spin_delta}, то есть рассмотрим три ветви, сходящихся в соединении, и обозначим $\sigma_\alpha (n)$ n-ый от соединения спин из ветви $\alpha$. Соединение в этой схеме более общего вида, оно называется $\Delta$-соединением. При необходимости, оно сводится к Т-соединению обнулением одной из констант связи.

Проведём теперь преобразование следующим образом. В каждой из ветвей сделаем своё преобразование Йордана-Вигнера. Тогда в каждой ветви взаимодействия будут соседними, а коммутационные соотношения будут соответствующие майорановским фермионом. Однако полученные таким образом частицы из разных ветвей будут коммутировать. Поэтому добавим в преобразование $i$-ой ветви матрицу $S^i$, квадрат каждой из которых равен единице, которые антикоммутируют друг с другом и коммутируют со всеми остальными спиновыми операторами. Поскольку ветви три, удобно выбрать в качестве $S^i$ матрицы Паули.

Модифицированное преобразование тогда выглядит следующим образом:
\begin{align}
    \gamma_{2i-1} &= \sigma^z (n) \sum\limits_{p=1}^{n-1} \sigma^x (p), & 
    \gamma_{2i} &= \sigma^y (n) \sum\limits_{p=1}^{n-1} \sigma^y (p).
\end{align}
Оно восстанавливает необходимые коммутационные соотношения и вид гамильтониана внутри ветвей. На соединении, однако, взаимодействие получается не квадратичным и включает в себя матрицы $S^i$.
\begin{equation}
    H_{junc} = - \sum\limits_{\alpha < \beta} J_{\alpha \beta} \sigma_\alpha^z (1) \sigma_\beta^z (1) = - i \sum\limits_{\alpha < \beta} J_{\alpha \beta} \gamma_\alpha (1) \gamma_\beta (1) \epsilon^{\alpha \beta \zeta} S^\zeta.
\end{equation}

Эту проблему можно решить так, как показано на рис. \ref{fig:spin_delta_central}, то есть сделать это взаимодействие трёхспиновым в исходной, спиновой системе. Помимо первых спинов цепочек туда теперь будет входить оператор дополнительного спина, который в дальнейшем будет называться "центральным"$ $. То есть взаимодействие будет выглядеть следующим образом:
\begin{equation}
    H_{junc} = - \frac{1}{2} \sum\limits_{\alpha \neq \beta \neq \zeta} J_{\alpha \beta} \sigma_\alpha^z (1) \sigma_\beta^z (1) S^\zeta = \frac{i}{2} \sum\limits_{\alpha \beta \zeta} J_{\alpha \beta} \gamma_\alpha (1) \gamma_\beta (1) \epsilon^{\alpha \beta \zeta}.
    \label{eq:junc_H}
\end{equation}

Таким образом, спиновая система с рисунка \ref{fig:spin_delta_central} является спиновым представлением системы майорановских фермионов, в которой удобно проводить переплетение.

Вопросы физической реализации этой системы следует оставить до части \ref{sec:spin_braiding}, в которой конструкция будет несколько изменена. 

\pagebreak

\subsection{Свойства спиновой системы}

Рассмотрим теперь, чему соответствуют краевые моды и процессы с ними на спиновом языке. Следующий материал частично описан в \cite{arrays} и \cite{main}.

Будем считать, что поперечное магнитное поле равно нулю внутри топологического участка и бесконечности - вне его. Это удобно с теоретической точки зрения, поскольку основное состояние принимает в таком случае простой вид, который будет указан далее, а с практической просто означает $h \ll J$ и $h \gg J$, что реализуемо. Изменение магнитного поля, приводящее к изменению размеров топологического участка, будем предполагать адиабатическим. Взаимодействие ферромагнитное, $J>0$ в гамильтониане \ref{eq:spin_H}.

В таком случае спины за пределами топологического участка оказываются эффективно "замороженными"$ $ вдоль оси $х$. Внутри же появляется изинговская цепочка без магнитного поля с эффективно свободными условиями на границах. Основное состояние такой системы двояко вырождено, базисными векторами можно считать состояния "все спины вверх"$ $, $\ket{\uparrow \dotsm \uparrow}$, и "все спины вниз"$ $, $\ket{\downarrow \dotsm \downarrow}$, где в кетах перечислены только спины топологического участка. Если температура достаточно низка, так что возбуждениями можно пренебречь, то все интересующие нас процессы происходят в двумерном подпространстве, построенном на этих векторах. 

Нетрудно показать, как влияет добавление к топологическому участку на его состояние. Для простоты рассмотрим двухспиновый гамильтониан
\begin{equation}
    H = -J \sigma_1^z \sigma_2^z - h \sigma_1^x.
    \label{eq:32_model_H}
\end{equation}
Его собственные значения $\lambda = \pm \sqrt{J^2 + h^2}$ и собственные функции
\begin{gather*}
    \psi_1 = \frac{\left( J + \sqrt{J^2 + h^2} \right) \uparrow + h \downarrow}{\sqrt{h^2 + \left(J + \sqrt{J^2 + h^2} \right)^2}} \uparrow, \\
    \psi_2 = \frac{ h \uparrow + \left( J + \sqrt{J^2 + h^2} \right) \downarrow}{\sqrt{h^2 + \left(J + \sqrt{J^2 + h^2} \right)^2}} \downarrow,
\end{gather*}
находятся очевидно. Из четырёх собственных функций эти две выбраны потому, что они отвечают наименьшей энергии, а температура предполагается низкой. В пределе $h \rightarrow \infty$ они становятся состояниями $\ket{\uparrow}_x \ket{\uparrow}_z$ и $\ket{\uparrow}_x \ket{\downarrow}_z$, где индекс обозначает, для какой матрицы Паули это состояние является собственным. Когда $h = 0$, они переходят в $\ket{\uparrow \uparrow}_z$ и $\ket{\downarrow \downarrow}_z$ соответственно.

Из предыдущего следует, что при "размораживании"$ $ спина, адиабатическом понижении поперечного магнитного поля от $\infty$ до $0$, произойдёт следующий переход:
\begin{equation*}
    \left( \frac{\ket{\uparrow} + \ket{\downarrow}}{\sqrt{2}} \right) (\alpha \ket{\uparrow} + \beta \ket{\downarrow}) \rightarrow \alpha \ket{\uparrow\uparrow} + \beta \ket{\downarrow\downarrow}.
\end{equation*}

Для спиновой цепочки это значит, что состояние топологического участка не меняется при передвижении его границ -  до тех пор, пока граница не проходит через центральный спин с трёхспиновым взаимодействием. Это оправдывает введение следующих обозначений.

Пусть $\ket{\uparrow} = \ket{\uparrow \dotsm \uparrow}$, $\ket{\downarrow} = \ket{\downarrow \dotsm \downarrow}$, где вновь указаны только спины топологического участка. Спиновый оператор $\tau$ действует на этих векторах. Здесь и далее топологический участок будет называться просто "кубитом"$ $. Движение границ в таких обозначениях не отражается на состоянии системы, что удобно. 

Осталось понять, как топологический участок проходит через центральный спин. Пусть, сначала, во взаимодействии участвует компонента $S^z$, а спин находится в состоянии $\ket{\uparrow}$ по $z$. Тогда гамильтониан на соединении \ref{eq:junc_H} станет эффективно двухспиновым, связывающим первые спины ветвей, с тем же коэффициентом, и просто не будет отличаться от всех остальных связей цепочки. Значит, при прохождении через него состояние кубита не изменится.

С другой стороны, если центральный спин находится в состоянии $\ket{\downarrow}$, верно всё то же, но взаимодействие меняет знак и становится антиферромагнитным. Интуиция подсказывает, а расчёты подтверждают, что в этом случае "размороженный"$ $ спин окажется перевёрнут относительно соседнего - \ref{eq:transition_states} иллюстрирует утверждение.

В момент перехода состояние системы является суперпозицией следующих векторов:
\begin{equation}
\begin{split}
    \ket{\dotsm \uparrow \uparrow} \ket{\uparrow}_s \ket{\uparrow \uparrow \dotsm}, \\
    \ket{\dotsm \downarrow \downarrow} \ket{\uparrow}_s \ket{\downarrow \downarrow \dotsm}, \\
    \ket{\dotsm \uparrow \uparrow} \ket{\downarrow}_s \ket{\downarrow \downarrow \dotsm}, \\
    \ket{\dotsm \downarrow \downarrow} \ket{\downarrow}_s \ket{\uparrow \uparrow \dotsm}.
\end{split}
\label{eq:transition_states}
\end{equation}
Видно, что переход действует на кубит как оператор $CNOT$, controlled not, контролируемый оператором $S^\alpha$. Например, в базисе ($\ket{\uparrow\uparrow}$, $\ket{\uparrow\downarrow}$, $\ket{\downarrow\uparrow}$, $\ket{\downarrow\downarrow}$), где первый спин - $S$, а второй - $\tau$,
\begin{equation}
CNOT_{S^z} = 
\begin{pmatrix}
    1 & 0 & 0 & 0 \\
    0 & 1 & 0 & 0 \\
    0 & 0 & 0 & 1 \\
    0 & 0 & 1 & 0
\end{pmatrix},
\end{equation}
причём контролирующий оператор указан в индексе.

\pagebreak

\subsection{Переплетение на спиновом языке} \label{sec:spin_braiding}
На введённом языке удобно рассматривать операцию переплетения, braiding. Будем работать в представлении Гейзенберга и исследовать эволюцию операторов $\tau$ и $S$. Нетрудно показать, что при прохождении через соединение со взаимодействием, включающим компоненту спина $S^\alpha$, операторы меняются следующим образом:
\begin{align}
    \tau^{y,z} &\rightarrow S^\alpha \tau^{y,z}, & \tau^x &\rightarrow \tau^x, \\
    S^{!\alpha} &\rightarrow \tau^x S^{!\alpha}, & S^\alpha &\rightarrow S^\alpha,
    \label{eq:spin_transformation}
\end{align}
где $!\alpha$ означает все компоненты, которые не $\alpha$.

Уже известно, как выглядит переплетение на майорановском языке. Проведём те же действия в спиновой системе. При этом топологический участок пройдёт последовательно через $S^z$, $S^x$, $S^y$. $\tau^x$ просто сохраняется всегда; каждый из $S^\alpha$ за время операции один раз сохраняется и два раза умножается слева на $\tau^x$, причём $\left( \tau^x \right)^2 = 1$. Так что изменятся только $\tau^y$ и $\tau^z$ составляющие.
\begin{equation}
    \tau^{y,z} \rightarrow S^z \tau^{y,z} \rightarrow \tau^x S^x S^z \tau^{y,z} \rightarrow S^y \tau^x S^x S^z \tau^{y,z} = -i \tau^x \tau^{y,z} = \pm \tau^{z,y}
\end{equation}
Таким образом, переплетение является поворотом на $\pi/2$ в пространстве спинового оператора кубита.

Можно заметить, что все изменения состояния кубита происходят в момент прохода через центральный спин. В отличие от майорановской системы, в которой пространственное расположение и неодномерность играют ключевую роль, в спиновой достаточно произвольным образом выполнить три "прогона"$ $, три прохода через нужные компоненты центрального спина, чтобы перевести систему в нужное конечное положение.

Возникает идея вернуться к одномерной цепочке. Рассмотрим систему, показанную на рис. \ref{fig:1d_spin_system}, и будем считать, что компонента спина во взаимодействии соединения может быть, при необходимости, изменена. Поскольку переплетение состоит из последовательных прогонов через три компоненты спина, в этом варианте конструкции оно будет выглядеть следующим образом: проведение кубита через $S^z$; изменение компоненты центрального спина, участвующей во взаимодействии, с $S^z$ на $S^x$; прогон топологического участка обратно через $S^x$; замена $S^x$ на $S^y$; ещё одно проведение кубита через спин. На рис. \ref{fig:1d_braiding} эта процедура показана поэтапно.

\begin{figure}
    \centering
    \includegraphics[width=\textwidth]{1d_spin_system}
    \caption{Одномерная спиновая система; компонента спина на соединении контролируется экспериментатором.}
    \label{fig:1d_spin_system}
\end{figure}

\begin{figure}
    \centering
    \captionsetup[subfigure]{justification=centering}
    \subfloat[]{
        \centering%
        \includegraphics[width=0.33\textwidth]{1d_spin_systemA}%
    }%
    \subfloat[]{
        \centering%
        \includegraphics[width=0.33\textwidth]{1d_spin_systemB}%
    }%
    \subfloat[]{
        \centering%
        \includegraphics[width=0.33\textwidth]{1d_spin_systemC}%
    }
    \caption{Переплетение в одномерной системе.}
    \label{fig:1d_braiding}
\end{figure}

Ранее было сказано о том, что если центральный спин находится в состоянии "вверх"$ $, соответствующем компоненте спина, участвующей во взаимодействии, то проведение кубита через соединение не меняет состояние кубита. Последний прогон в данной схеме происходит через компоненту $S^y$; если бы в этот момент спин находился в состоянии "вверх"$ $ по $y$, то этот прогон можно было бы не выполнять. Зная, как изменяются компоненты спина в процессе прогонов (\ref{eq:spin_transformation}), можно понять, что к моменту третьего проведения кубита через спин $y$-компонента последнего совпадает с той, что была в начале. Значит, если до начала операции переплетения центральный спин был в состоянии $\ket{\uparrow}_y$, то третий прогон можно не осуществлять.

Это полезно в первую очередь потому, что общее время операции уменьшается в полтора раза. Шумы и малое время когерентности ограничивают возможное количество операций над кубитом, поэтому уменьшение времени выполнения отдельной операции - это положительный результат. Кроме того, в такой схеме кубит после переплетения оказывается в том же месте цепочки, в котором он был в начале, - нет побочного эффекта смещения топологического участка относительно центрального спина, что может быть или не быть желаемым поведением.

\pagebreak

\section{Шумы и поправки}

В этой части дипломной работы будут приведены качественные и количественные оценки влияния шумов и неточностей оборудования на спиновую систему. Предполагая их малыми, вычислим первые порядки всех эффектов и, по возможности, сравним их.

\subsection{Обозначения}

Будем предполагать, что до начала переплетения центральный спин и кубит расцеплены, причём кубит обладает какой-либо допустимой матрицей плотности, а спин, в зависимости от схемы, находится либо в произвольном состоянии, либо направлен вверх по $y$.
\begin{gather}
    \rho_\tau^{init} = \frac{1}{2} (1 + \vec{r}\vec{\tau}) = \frac{1}{2}
    \begin{pmatrix}
        1 + z & x + i y \\
        x - i y & 1 - z
    \end{pmatrix}, \quad \abs{\vec{r}} \le 1,
    \\
    \rho_s^{init} = \frac{1}{2} \begin{pmatrix}
        1 & i \\
        -i & 1
    \end{pmatrix},
    \\
    \rho^{init} = \rho_\tau^{init} \otimes \rho_s^{init}.
\end{gather}

Выполняемые преобразования сводятся к CNOT, причём "контролирующими"$ $ являются разные компоненты спина $S$. Получить эти матрицы можно как непосредственно рассматривая действие на базисные векторы, так и поворотом $CNOT_{S^z}$.
\begin{gather}
    U_3 = CNOT_{S^z} = \begin{pmatrix}
        1 & 0 & 0 & 0 \\
        0 & 1 & 0 & 0 \\
        0 & 0 & 0 & 1 \\
        0 & 0 & 1 & 0
    \end{pmatrix},
    \displaybreak[0]\\
    U_1 = CNOT_{S^x} = \frac{1}{2} 
    \begin{pmatrix}
        1 & 1 & 1 & -1 \\
        1 & 1 & -1 & 1 \\
        1 & -1 & 1 & 1 \\
        -1 & 1 & 1 & 1
    \end{pmatrix},
    \displaybreak[0]\\
    U_2 = CNOT_{S^x} = \frac{1}{2} 
    \begin{pmatrix}
    1 & 1 & i & -i \\
    1 & 1 & -i & i \\
    i & -i & 1 & 1 \\
    -i & i & 1 & 1
    \end{pmatrix}.
\end{gather}

После выполнения переплетения в схеме с двумя прогонами
\begin{gather}
    \rho^{final} = U_1 U_3 \rho^{init} (U_1 U_3)^\dagger = \rho^{final}_\tau \otimes \rho^{final}_s,
    \\
    \rho^{final}_\tau = \frac{1}{2}
    \begin{pmatrix}
        1 - y & x + i z \\
        x - i z & 1 + y
    \end{pmatrix} = \frac{1}{2} (1 + x \tau_x + z \tau_y - y \tau_z), 
    \label{eq:tau_final}
    \\
    \rho^{final}_s = \rho^{init}_s.
\end{gather}
К этому результату в следующих параграфах будут найдены различные поправки. Регулярно будет получаться так, что кубит и спин из-за неидеальности системы в конце операции останутся зацепленными; в таком случае будет приводиться в том числе и результат, усреднённый по состояниям центрального спина. 
Такой подход хорошо характеризует систему, если кубит и данный спин в дальнейшем не будут взаимодействовать - например, для дальнейших операций будет использован другой спин.

\subsection{Предположения}
Как уже было упомянуто в части \ref{sec:spin_braiding} {\color{red}НЕ БЫЛО!!!}, к центральному спину предъявляются дополнительные технические требования. 
Из-за этого его время корреляции, скорее всего, будет меньше времён корреляции всех остальных спинов цепочки. 
Поэтому в первую очередь будет рассматриваться то, как действие шумов на центральный спин влияет на состояние кубита после переплетения. 
Это действие удобно рассматривать в несколько этапов. 
Есть поперечный и продольный шумы, а топологический участок два раза проходит через центральный спин (в схеме с двумя прогонами).
Согласно этому разделим вычисления на три этапа: два, в которых спин находится внутри топологического участка, и один, в котором он находится вне. Вычисления, проведённые на каждом этапе потом можно сложить, потому что первый порядок теории возмущений линеен.

Кроме того, будут исследованы погрешности, вносимые неточностью манипуляций, такие как малая ошибка при повороте спина между прогонами и малое отличие начального состояние спина от $\ket{\uparrow}_y$.

\subsection{Начальное состояние центрального спина в схеме с тремя прогонами}

Из сравнения процесса переплетения в майорановской и спиновой системах заранее очевидно, что в схеме с тремя прогонами центральный спин может находиться в произвольном чистом состоянии, это не должно повлиять на результат операции.
Интересно, что даже если спин изначально находился в смешанном состоянии, то есть был зацеплен с внешней средой, кубит не переходит в смешанное состояние в результате переплетения. 
Здесь имеются в виду только различные начальные состояния спина; шумы, как и неточности оборудования, предполагаются отсутствующими.

Это можно проверить непосредственным вычислением, посчитав, что
\begin{equation}
    U_2 U_1 U_3 (\rho_\tau \otimes \rho_s) (U_2 U_1 U_3)^\dagger = \rho^{final}_\tau \otimes \rho_s
    \label{eq:u213_general},
\end{equation}
где $\rho^{final}_\tau$ из $\ref{eq:tau_final}$, а $\rho_s$ произвольная допустимая.

Впрочем, это можно понять и так. В гейзенберговском представлении матрица плотности вообще не меняется; законы преобразования операторов не зависят от того, является состояние чистым или смешанным; а для чистых состояний, как было указано выше, результат очевиден из сравнения с майорановской системой.

Этот результат имеет качественное значение: несмотря на недостатки схемы с тремя прогонами, нет необходимости делать хоть какие-то операции со спином до начала переплетения, это не повлияет на результат. 
В схеме с двумя прогонами спин должен находиться в особом начальном состоянии, что служит дополнительным источником ошибок.

\subsection{Спин внутри топологического участка и поперечный шум}

Другим качественным наблюдением является то, что влияние поперечного шума на центральный спин значительно ослаблено, пока он находится внутри топологического участка. 
Внимательный взгляд на волновые функции \ref{eq:transition_states} основного состояния системы в таком положении показывает, что в этом основном состоянии нет пар функций, получающихся друг из друга просто переворотом центрального спина: есть только пары, получающиеся переворотом его и одной из частей топологического участка, слева или справа от самого спина.

Это означает, что переходы, происходящие под влиянием поперечного шума, появляются только в $(1 + \min(m,n))$-ом порядке теории возмущений, где $m$ - число спинов топологического участка слева от спина, $n$ - оно же справа. 
То есть, как минимум, не в первом, пока спин находится внутри топологического участка.

\subsection{Погрешности оборудования}

Пусть начальное состояние спина немного отличается от $\ket{\uparrow}_y$. Рассмотрим влияние этого отличия на результат в первом порядке. 
Точно, пусть волновая функция спина
\begin{equation}
    \psi = \ket{\uparrow} + \epsilon \ket{\downarrow},
\end{equation}
причём поправка к волновой функции из-за нормировки второго порядка, а соответствующее изменение матрицы плотности
\begin{equation}
    \delta\rho^{init}_s = \frac{\epsilon}{2}
    \begin{pmatrix}
        1 & 0 \\
        0 & -1 
    \end{pmatrix}.
\end{equation}

Вычисляя $U_1 U_3 \rho^{init} (U_1 U_3)^\dagger$, найдём поправку к результату
\begin{equation}
    \delta \rho^{final} = \frac{\epsilon}{4}
    \begin{pmatrix}
        x & 1 & z & i y \\
        1 & x & - i y & -z \\
        z & i y & -x & -1 \\
        - i y & - z & -1 & -x
    \end{pmatrix}
\end{equation}
и её частичный след по состояниям центрального спина
\begin{equation}
    \delta \rho^{final}_\tau = 
    \begin{pmatrix}
        0 & 0 \\
        0 & 0
    \end{pmatrix}.
\end{equation}

Несмотря на то, что первый порядок поправки к результату более не позволяет разбивать всю матрицу плотности на тензорное произведение матриц плотностей кубита и спина, после усреднения по спиновым состояниям поправка оказывается тождественно равной нулю. Она возникнет только во втором порядке.

Также ошибка может возникнуть во время поворота спина между прогонами. Поворот происходит вокруг оси $y$ на $\pi/2$. Пусть угол оказался смещён на $\delta \phi$. Рассматривая $U_1$ как
\begin{gather}
    U_1 = R^\dagger U_3 R,
    \\
    R = \frac{1}{\sqrt{2}}
    \begin{pmatrix}
        1 & 0 & 1 & 0 \\
        0 & 1 & 0 & 1 \\
        -1 & 0 & 1 & 0 \\
        0 & -1 & 0 & 1
    \end{pmatrix},
    \\
    \delta R = \frac{\delta \phi}{\sqrt{2}}
    \begin{pmatrix}
        -1 & 0 & 1 & 0 \\
        0 & -1 & 0 & 1 \\
        -1 & 0 & -1 & 0 \\
        0 & -1 & 0 & -1
    \end{pmatrix},
\end{gather}
где $\delta R$, поправка к матрице поворота, получена разложением в ряд Тейлора по $\delta \phi$.

Соответственно, для изменения результата получается
\begin{multline}
    \delta \rho^{final} = \left( \delta R^\dagger U_3 R U_3 + R^\dagger U_3 \delta R U_3 \right) \rho^{init} (U_1 U_3)^\dagger \\
    + U_1 U_3 \rho^{init} \left( \delta R^\dagger U_3 R U_3 + R^\dagger U_3 \delta R U_3 \right)^\dagger.
\end{multline}
Выражение, в котором уже взят частичный след по спиновым состояниям, выглядит так:
\begin{equation}
    \delta \rho^{final} = 
    \begin{pmatrix}
        -z & - i y \\
        i y & z
    \end{pmatrix}
\end{equation}.

\subsection{Продольный шум, спин внутри топологического участка}

В процессе переплетения спин находится внутри топологического участка два раза: когда во взаимодействии стоит $S^z$ и $S^x$.
За это время между состояниями $\ket{\uparrow}_\alpha$ и $\ket{\downarrow}_\alpha$, где $\alpha = z, x$, набегает случайная фаза.
Следует обратить внимание на то, что во втором случае, когда речь идёт об $S^x$, шум остаётся продольным, но уже вдоль оси $x$. 
Это связано с тем, что взаимодействие имеет вид $\tau^z S^x \tau^z$, и от поперечного $x$ шума система защищена так, как описано ранее. 
Тот же аргумент становится очевидным в шрёдингеровском представлении: если в гейзенберговском менялся оператор спина, участвующий во взаимодействии, то в шрёдингеровском взаимодействие не меняется вообще, поворачивается лишь спин.
Поэтому не является удивительным то, что поправка, набирающаяся за эти прогоны, одинакова.

Представляя $CNOT$ в виде
\begin{equation}
    U_3 = CNOT_{S^z} = 
    \begin{pmatrix}
        \mathrm{e}^{i \phi} & 0 & 0 & 0 \\
        0 & \mathrm{e}^{i \phi} & 0 & 0 \\
        0 & 0 & 0 & \mathrm{e}^{-i \phi} \\
        0 & 0 & \mathrm{e}^{-i \phi} & 0
    \end{pmatrix},
\end{equation}
вычисляя $\rho^{final}$, усредняя его по реализациям шума, разлагая получившиеся экспоненты до первого порядка и усредняя по состояниям центрального спина, получаем
\begin{equation}
    \rho^{final}_\tau = \frac{t}{2T_2}
    \begin{pmatrix}
        y & - i z \\
        i z & - y
    \end{pmatrix}
\end{equation}
где $t$ - это время прогона, $T_2$ - время поперечной релаксации. 
\pagebreak

\bibliographystyle{unsrt}
\bibliography{diploma}

\end{document}