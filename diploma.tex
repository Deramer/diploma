\documentclass[a4paper,12pt]{article}

%%	Fonts
\usepackage{cmap}
\usepackage{mathtext}
\usepackage{hyperref}
\usepackage[T2A]{fontenc}
\usepackage[utf8]{inputenc}
\usepackage[english,russian]{babel}
\usepackage[scaled]{helvet}
%\usepackage{fullpage}

% 	Math packages
\usepackage{amsmath,amsfonts,amssymb,amsthm,mathtools}
\usepackage{icomma}
\usepackage{euscript}
\usepackage{mathrsfs}
\usepackage{tensor}
\usepackage{physics}
\usepackage{stackrel}
% 	My math operartors
%\mathtoolsset{showonlyrefs=true}
\newcommand{\bigo}[1]   {\,\ensuremath{\mathop{}\mathopen{}\mathcal{O}\mathopen{}\displaystyle\left(#1\right)}}
\newcommand{\smallo}[1]	{\,\scriptstyle\mathop{}\mathopen{}\mathcal{O}\mathopen{}\displaystyle\left(#1\right) }
\DeclareMathOperator{\arcsinh}{arcsinh}
\DeclareMathOperator{\arctanh}{arctanh}
\newcommand*{\hm}[1]{#1\nobreak\discretionary{} {\hbox{$\mathsurround=0pt #1$}}{}}

%%	Graphics
\usepackage{graphicx}
\graphicspath{{images/}}
\setlength\fboxsep{3pt}
\setlength\fboxrule{1pt}
\usepackage{wrapfig}
% 	Tables
\usepackage{array,tabularx,tabulary,booktabs}
\usepackage{longtable}
\usepackage{multirow}
\usepackage{caption}
%\usepackage{subcaption}
\usepackage{subfig}
%	Itemize
\usepackage{enumitem}
\setlist[itemize]{topsep=0pt, leftmargin=0.2in}

%% 	Theorems
\theoremstyle{plain} % Это стиль по умолчанию, его можно не переопределять.
 \newtheorem{theorem}{Теорема}[section]
 \newtheorem{proposition}{Утверждение}
%\newtheorem{proposition}[theorem]{Утверждение}
 \newtheorem{lemma}{Лемма}
\theoremstyle{definition} % "Определение"
 \newtheorem{definition}{Определение}[section]
 \newtheorem{corollary}{Следствие}[theorem]
 \newtheorem{problem}{Задача}[section]
\theoremstyle{remark} % "Примечание"
 \newtheorem*{solution}{Решение}
\renewcommand\qedsymbol{$\blacksquare$}
\newcommand{\proofbegin}{\ensuremath{\blacktriangle}\nopunct}
\newcommand{\contbegin}{\footnotesize{\textcircled{\scriptsize!}}\normalsize \ }
\newcommand{\contend}{\ensuremath{\otimes}}
\newcommand{\vect}[1]{\ensuremath{\overrightarrow{#1}}}

%%	Programming
\usepackage{etoolbox} % логические операторы

%%	Page
\usepackage[14pt]{extsizes} % Возможность сделать 14-й шрифт
\usepackage{geometry} % Простой способ задавать поля
	\geometry{top=20mm}
	\geometry{bottom=20mm}
	\geometry{left=22mm}
	\geometry{right=22mm}
	\geometry{bindingoffset=0mm}
%\linespread{0.01}
%\usepackage{mathpazo}
%usepackage{fancyhdr} % Колонтитулы
% 	\pagestyle{fancy}
% 	\renewcommand{\headrulewidth}{.1mm}  % Толщина линейки, отчеркивающей верхний колонтитул
% 	\lfoot{}
% 	\rfoot{}
% 	\rhead{}
% 	\chead{}
% 	\lhead{}
 	% \cfoot{Нижний в центре} % По умолчанию здесь номер страницы
\usepackage{setspace} % Интерлиньяж
\onehalfspacing % Интерлиньяж 1.5
%\doublespacing % Интерлиньяж 2
%\singlespacing % Интерлиньяж 1
%\linespread{0.01}
\usepackage{lastpage} % Узнать, сколько всего страниц в документе.
\usepackage{soul} % Модификаторы начертания
%\usepackage{hyperref}
\usepackage[usenames,dvipsnames,svgnames,table,rgb]{xcolor}
\hypersetup{					% Гиперссылки
    unicode=true,           	% русские буквы в раздела PDF
    pdftitle={},   				% Заголовок
    pdfauthor={Nestyuk Arseniy},	% Автор
    pdfsubject={},      		% Тема
    pdfcreator={Nestyuk Arseniy}, 	% Создатель
    pdfproducer={}, 			% Производитель
    pdfkeywords={Qubit} {Quantum physics} {Majorana}, % Ключевые слова
    colorlinks=true,       		% false: ссылки в рамках; true: цветные ссылки
    linkcolor=red,          	% внутренние ссылки
    citecolor=green,        	% на библиографию
    filecolor=magenta,      	% на файлы
    urlcolor=cyan           	% на URL
}
\usepackage{multicol} % Несколько колонок

% 	Titel
\author{}
\title{}
\date{\today}

\graphicspath{ {images/} }


\begin{document}
    
\begin{titlepage}
    \begin{center}
        Министерство образования и науки Российской Федерации
        \linebreak
        
        Федеральное государственное автономное
        
        образовательное учреждение высшего образования 
        
        «Московский физико-технический институт 
        
        (государственный университет)»
        \linebreak
        
        Факультет общей и прикладной физики
        
        Кафедра проблем теоретической физики
        \vspace{100pt}
        
        \textbf{\LARGE Майорановские кубиты в джохефсоновских изинговских цепочках}
        \linebreak
        
        Выпускная квалификационная работа бакалавра
     \end{center}
     \vspace{90pt}
     
     \begin{flushright}
         \begin{tabular}{cl}
         &Выполнил:\\
         &студент 322 группы\\         
         &Нестюк А.П.\\
         &\\
         &Научный руководитель:\\
         &д.ф-м.н., проф. Махлин Ю.Г.
        \end{tabular}
    \end{flushright}
    \vspace{60pt}
    
    \begin{center}
        Черноголовка, 2017
    \end{center}
 \end{titlepage}   

\tableofcontents
\pagebreak

\section{Введение}
В последнее время большой интерес вызывают возможности реализации кубитов на основе так называемых майорановских фермионов. Реализация таких кубитов и системы управления ими может оказаться ключевым фактором в построении квантового компьютера, способного решать реальные задачи.

Впервые майорановские фермионы были рассмотрены Китаевым в \cite{kitaev}. Он рассмотрел фермионную цепочку и провёл над ней преобразование Майораны, показав, что при выполнении определённых соотношений между коэффициентами гамильтониана появляется возбуждение с нулевой энергией, эффективно составленное из фермионных операторов, находящихся на разных концах цепочки. Физические, локальные шумы меняют энергию этого возбуждения экспоненциально слабо по длине цепочки. Потеря когерентности из-за влияния шумов является одной из основных проблем создания кубитов, и данная система практически полностью решает этот вопрос.

К сожалению, решая один вопрос, майорановские фермионы ставят новые. Во-первых, их техническая реализация ещё далека от совершенства. Самой многообещающей основой для создания таких систем сейчас считаются p-волновые сверхпроводники {\color{red}ссылка}, и в них получены только признаки существования таких фермионов. Об управлении ими речь пока не идёт.

Во-вторых, строго доказано, что нельзя выполнить произвольный квантовый алгоритм, не нарушив защиту майорановских фермионов. В науке о квантовых алгоритмах есть понятие "универсального набора операций"$ $. Если такой набор операций реализуем на данной физической системе, то возможно выполнение произвольного действия над данной системой, и тем выполним любой алгоритм, в противном случае это невозможно. Над майорановскими фермионами реализовать универсальный набор операций не получится.

Тем не менее, интерес к майорановским фермионам не угас. Технические сложности обычно преодолимы, и даже если не получится производить над такими кубитами нужные операции, их можно будет использовать в качестве долговременной памяти. 

Более того, оказались простимулированы исследования, связанные с прочими системами, которые можно описывать на языке майорановских фермионов. Одной из таких систем является спиновая цепочка, над которой совершено преобразование Йордана-Вигнера. Очевидно, получившуюся фермионную систему можно свести к системе майорановских фермионов. Однако, поскольку преобразования Йордана-Вигнера нелокально, обычные локализованные шумы спиновой системы становятся нелокальными шумами в фермионной системе, из-за чего теряется защита от шумов - основная причина интереса. Тем не менее, поскольку без разрушения защиты невозможно выполнять квантовые алгоритмы, можно нарушать её и в такой форме. Поскольку спиновые системы на основе джозефсоновских кубитов изучены относительно хорошо, а времена их когерентности позволяют проводить с ними эксперименты, можно уже сейчас проверить, как соотносится теория с практикой. Кроме того, возможно, что майорановские кубиты на сверхпроводниках будут использоваться в основном как память компьютера, а спиновые системы, полностью находящиеся под нашим контролем, - как способ проводить операции и хранить данные небольшое количество времени (процессор в терминах современных компьютеров).

В работе \cite{main} описаны теоретические основы и элементы практической реализации подобных спиновых систем. Единственная доступная операция над майорановскими фермионами требует выхода за пределы одномерной цепочки, что представляет дополнительные сложности в момент выполнения преобразования Йордана-Вигнера. Дальнейшие возникающие затруднения и указанные в статье методы их устранения будут приведены ниже.

Целью данной работы является более подробное изучение спиновой системы, описанной в статье \cite{main}. Будут рассмотрены различные погрешности, включая влияние различных шумов на состояние кубита и фактический размер майорановских фермионов. Будут сделаны некоторые общие выводы относительно защищённости этой системы от случайных шумов.

{\color{red}А здесь можно привести содержание описания (в главе х рассказывается про у...)}

% Moving to next section
% Majorana modes in fermionic systems

\pagebreak

\section{Майорановские моды в фермионных системах}

Основой для всего дальнейшего служат статьи \cite{kitaev}, в которой вводится понятие майорановских фермионов, и \cite{braiding}, где предлагается методика выполнения операции над ними.

\subsection{Модель Китаева}
Следуя \cite{kitaev}, рассмотрим фермионную цепочку со следующим гамильтонианом:
\begin{equation}
    H = \sum\limits_j -\omega (a_j^\dagger a_{j+1} + a_{j+1}^\dagger a_j - a_j a_{j+1} - a_j^\dagger a_{j+1}^\dagger) - \mu \left( a_j^\dagger a_j - \frac{1}{2} \right)
\end{equation}
(строго говоря, в гамильтониане Китаева коэффициенты перед $a_j^\dagger a_{j+1}$ и $a_j a_{j+1}$ разные, но для простоты здесь они положены одинаковыми).

Майорановскими операторами называется набор операторов
\begin{align}
\gamma_{2j-1} &= a_j + a_j^\dagger, & \gamma_{2j} &= i\left( a_j^\dagger - a_j \right),
\end{align}
удовлетворяющий соотношениям
\begin{align}
\gamma_i^2 &= 1, & \{\gamma_i, \gamma_j\} = 2 \delta_{ij},
\end{align}
причём чётные и нечётные индексы более неразличимы.

Гамильтониан после преобразования имеет вид
\begin{equation}
H = -i\mu/2 \sum\limits_{j=1}^N \gamma_{2j-1} \gamma_{2j} + i \omega \sum\limits_{j=1}^{N-1} \gamma_{2j} \gamma_{2j+1}.
\label{eq:main_hamiltonian}
\end{equation}
Нетрудно заметить, что в предельных случаях, когда либо $\mu$, либо $\omega = 0$, получаются два принципиального разных гамильтониана,
\begin{align}
H &= -i\mu/2 \sum\limits_{j=1}^N \gamma_{2j-1} \gamma_{2j}, & H &= i \omega \sum\limits_{j=1}^{N-1} \gamma_{2j} \gamma_{2j+1}.
\end{align}
Во втором отсутствуют члены $\gamma_1$ и $\gamma_{2L}$, которые вместе составляют одно возбуждение с нулевой энергией. Предполагая сохранение чётности, учитывая то, что квадрат майорановского оператора равен 1, и рассматривая первый порядок теории возмущений, получим, что шум, расщепляющий этот уровень, должен быть произведением операторов $a_1$, $a_L$ и их сопряжённых. Такой шум, очевидно, нелокален, и потому нефизичен. Отсюда делается вывод о так называемой топологической защите этого уровня от шума. Слово "топологический"$ $ связано с тем, что майорановские моды локализованы на краях цепочки.

В первом гамильтониане возбуждения с нулевой энергией нет. Как показывает точный расчёт, когда оба коэффициента в гамильтониане отличны от нуля, наличие такого возбуждения зависит от отношения коэффициентов $\mu$ и $\omega$: когда $\mu/\omega > 2$, его нет, в противном случае есть. Причём когда возбуждение есть, оно затухает вглубь образца с экспоненциальной скоростью.

Видно, что образуются две фазы, причём в одной из них на границе "живут"$ $ майорановские фермионы, также называемые краевыми модами. Такую фазу будет далее называть топологической.


\subsection{Переплетение майорановских фермионов}

Поскольку майораноские моды не входят в гамильтониан, возникает проблема управления ими, а без контролируемого изменения состояния кубита невозможны квантовые вычисления.

В \cite{braiding} предлагается операция, названная braiding. Это слово в данной дипломной работе будет переводиться как "переплетение"$ $. Идея состоит в том, что если поменять местами майорановские фермионы, то состояние системы изменится. Это легко понять, введя оператор чётности
\begin{equation}
T = \prod\limits_{j=1}^{2L} (-i \gamma_j).
\end{equation}
Этот оператор коммутирует с гамильтонианом \ref{eq:main_hamiltonian}. В самом деле, в гамильтониан входят только чётные по числу майорановских фермионов слагаемые, а в оператор чётности - каждый из таких фермионов один раз. Учитывая, что различные фермионы антикоммутируют, а сами с собой - коммутируют, получим
\begin{align}
\gamma_n T &= - T \gamma_n & \gamma_n \gamma_m T = T \gamma_n \gamma_m.
\end{align}

Если теперь поменять местами два фермионных оператора, оператор чётности изменит знак:
\begin{multline}
T_{nm} = (-i)^{2N} \gamma_1 \dotsm \gamma_n \dotsm \gamma_m \dotsm \gamma_{2N} = 
    (-i)^{2N} (-1)^{m-n-1} \gamma_1 \dotsm \gamma_n \gamma_m \dotsm \gamma_{2N} \\ =
    (-i)^{2N} (-1)^{m-n} \gamma_1 \dotsm \gamma_m \gamma_n \dotsm \gamma_{2N} = 
     - (-i)^{2N} \gamma_1 \dotsm \gamma_m \dotsm \gamma_n \dotsm \gamma_{2N} = 
     T_{mn} = - T_{nm}.
\end{multline}
Если совершается физическое преобразование, которое в реальном пространстве меняет местами майорановские моды и сохраняет при этом чётность, то операторы (в гейзенберговском представлении) не могут просто переходить друг в друга: один из них меняет знак, чтобы сохранить чётность. Соответственно этому изменится и состояние системы.

В одномерной цепочке невозможно переставить фермионы, не проведя их друг через друга. В \cite{braiding} предлагается система, в которой это возможно. Исследуем изменения относительно предыдущего пункта поэтапно.

Сначала нужно рассмотреть фермионную цепочку, отношение коэффициентов гамильтониана которой показанно на рисунке \ref{fig:two_phases}. Внутри каждого из участков коэффициенты гамильтониана постоянны, при этом в среднем участке отношение коэффициентов такого, что он находится в топологической фазе, а другие участки - в фазе без фермионов.

\begin{figure}
    \centering
    \begin{minipage}{.5\textwidth}
        \centering
        \includegraphics[width=\linewidth]{two_phases}
        \captionsetup{width=0.9\textwidth}
        \captionof{figure}{Зависимость отношения коэффициентов гамильтониана от номера узла; горизонтальная линия показывает границу фаз.}
        \label{fig:two_phases}
    \end{minipage}%
    \begin{minipage}{.5\textwidth}
        \centering
        \includegraphics[width=\linewidth]{T_junction}
        \captionsetup{width=0.9\textwidth}
        \captionof{figure}{T-соединение; точками показаны фермионы, линиями - их взаимодействие.}
        \label{fig:t_junction}
    \end{minipage}
\end{figure}

В такой системе майорановские фермионы также будут присутствовать; при конечном отношении коэффициентов возбуждение будет экспоненциально затухать не только внутрь топологической фазы, но и "наружу"$ $, в обычную фазу.

Обладая контролем над коэффициентами гамильтониана, можно изменять их отношение так, что граница фаз будет смещаться. При адиабатическом выполнении данной операции состояние майорановских фермионов не будет меняться. Это означает, что мы можем двигать топологический участок в произвольное место цепочки.

Наконец, можно представить себе соединение двух фермионных цепочек. Простейший вариант, представленный на рисунке \ref{fig:t_junction} и называемый T-соединением, является обычной фермионной цепочкой, один из узлов которой взаимодействует с крайним фермионом другой цепочки. В такой системе тоже существуют две фазы по отношению коэффициентов гамильтониана, и на границе между ними находятся майорановские фермионы. Соответственно, границы между фазами в такой конструкции можно двигать в том числе и в другие участки соединения.

\begin{figure}
    \centering
    \subfloat{%
        \centering
        \includegraphics[width=0.25\textwidth]{braidingA}}%
    \subfloat{%
        \centering
        \vspace{10pt}
        \includegraphics[width=0.25\textwidth]{braidingB}}%
    \subfloat{%
        \centering
        \includegraphics[width=0.25\textwidth]{braidingC}}%
    \subfloat{%
        \centering
        \includegraphics[width=0.25\textwidth]{braidingD}}%
    \caption{Braiding, переплетение майорановских мод в Т-соединённых фермионных цепочках, \cite{braiding}.}
    \label{fig:braiding}
\end{figure}

Теперь можно понять, как происходит перестановка майорановских мод. Процесс представлен на рисунке \ref{fig:braiding}. В вышеописанной системе одна из мод уводится в нижнее ответвление, другая переводится на исходное место первой, после чего первая передвигается туда, где в начале была вторая. Все процессы осуществляются адиабатически и с сохранением чётности.

% Moving to next section
% Spin system

\pagebreak

\section{Спиновая система}

В данной части будет кратко рассмотрено содержание \cite{main}. Будет показано, как можно совершить переход от спиновой системы к фермионной, с учётом того факта, что цепочка не является одномерной и обычное преобразование Йордана-Вигнера не даёт гамильтониана с соседним взаимодействием. Далее будет рассмотрено переплетение на спиновом языке.

\subsection{Модифицированное преобразование Йордана-Вигнера}

Обычное преобразование Йордана-Вигнера (здесь переводящее спиновую систему на язык майорановских фермионов),
\begin{align}
\gamma_{2i-1} &= \sigma^z (n) \sum\limits_{p=1}^{n-1} \sigma^x (p), & 
\gamma_{2i} &= \sigma^y (n) \sum\limits_{p=1}^{n-1} \sigma^y (p),
\end{align}
превращает изинговскую цепочку с поперечным магнитным полем
\begin{equation}
H = - \sum\limits_{n=1}^N h(n) \sigma^x(n) - J \sum\limits_{n=1}^{N-1} \sigma^z(n) \sigma^z(n+1)
\end{equation}
в майорановскую цепочку с квадратичным гамильтонианом и соседним взаимодействием
\begin{equation}
H = -i \sum\limits_{n=1}^N h(n) \gamma(2n-1) \gamma(2n) + i J \sum\limits_{n=1}^{N-1} \gamma(2n) \gamma(2n+1).
\end{equation}

Для этого преобразования важно уметь последовательно перенумеровывать спины. В случае, когда есть три ветки (рисунок \ref{fig:t_junction}, например), невозможно перенумеровать спины таким образом, чтобы после преобразования получилась майорановская цепочка, в которой взаимодействуют только соседи. Поэтому приходится менять вид преобразования.

\begin{figure}
    \centering
    \captionsetup{width=0.45\textwidth}
    \subfloat[Нумерация спинов в спиновой системе, необходимая для проведения преобразования Йордана-Вигнера.]{
        \centering%
        \includegraphics[width=0.5\textwidth]{spin_delta}%
        \label{fig:spin_delta}%
    }%
    \subfloat[Спиновая система с центральным спином и его вкладом во взаимодействие первых спинов цепочек.]{
        \centering%
        \includegraphics[width=0.5\textwidth]{spin_delta_central}%
        \label{fig:spin_delta_central}%
    }
    \caption{$\Delta$-соединение цепочек.}
\end{figure}

Перенумеруем спины так, как указано на рисунке \ref{fig:spin_delta}.
\pagebreak

\bibliographystyle{unsrt}
\bibliography{diploma}

\end{document}